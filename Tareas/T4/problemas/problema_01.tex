\documentclass[../main.tex]{subfiles}

\begin{document}
\begin{problema}[10]
	Considerando un problema con solo una dimensión espacial, muestre que

	\begin{enumerate}
		\item El operador \(\odv{}{x}\) no es un operador hermiteano.
		\item El operador de momento es un operador hermiteanos, es decir,
		      \begin{equation*}
			      \avg{\op{p}}^{*} = \avg{\op{p}}.
		      \end{equation*}
	\end{enumerate}

	\startsolution

	\section{Inciso (a)}

	Para determinar  que \(\odv{}{x}\) no es un operador hermiteano, debemos verificar que

	\begin{equation}
		\avg{\odv{}{x}} \neq \avg{\odv{}{x}}^{*}.
	\end{equation}

	Por un lado, tenemos que el lado izquierdo de la ecuación es

	\begin{align}
		\avg{\odv{}{x}} & = \matrixel{\psi}{\odv{}{x}}{\psi} = \int \odif[sep-end=\medspace]{x} \psi^{*}\Bigl(\odv{}{x}\Bigr)\psi,\nonumber \\
		\avg{\odv{}{x}} & = \int \odif[sep-end=\medspace]{x} \psi^{*}\odv{\psi}{x}.\label{eq:avg-psi}
	\end{align}

	Si calculamos el conjugado de la \zcref{eq:avg-psi},

	\begin{align*}
		\avg{\odv{}{x}}^{*} & = \Biggl[\int \odif[sep-end=\medspace]{x} \psi^{*}\Bigl(\odv{}{x}\Bigr)\psi\Biggr]^{*},\nonumber \\
		                    & = \int \odif[sep-end=\medspace]{x} \psi \odv{\psi^{*}}{x}.
	\end{align*}

	Integrando por parte con \(u = \psi\) y \( \odif{v} = \odv{\psi^{*}}{x} \odif{x}\),

	\begin{align*}
		\avg{\odv{}{x}}^{*} & = \psi \int \odif[sep-end=\medspace]{x} \odv{\psi^{*}}{x} - \int \odif[sep-end=\medspace]{x} \odv{\psi}{x} \psi^{*},
		                    & = \psi \Bigl(\psi\Bigr)_{-\infty}^{\infty} - \int \odif[sep-end=\medspace]{x} \odv{\psi}{x} \psi^{*}.
	\end{align*}

	Por tareas anteriores sabemos que no hay función de onda en \(\pm \infty\), por lo
	que el primer término es cero,

	\begin{equation*}
		\avg{\odv{}{x}}^{*} = - \int \odif[sep-end=\medspace]{x} \odv{\psi}{x} \psi^{*}.
		\label{eq:avg-psi-star}
	\end{equation*}

	Así, de las \zcref{eq:avg-psi,eq:avg-psi-star} concluimos que
	\(\odv{}{x}\) no es hermiteano.

	\section{Inciso (b)}

	El operador \(\op{p}\) en una dimensión es

	\begin{equation*}
		\op{p} = \dfrac{\hbar}{i}\odv{}{x}.
	\end{equation*}

	Entonces,

	\begin{align}
		\avg{\op{p}} & = \int \odif[sep-end=\medspace]{x} \psi^{*}\Bigl(\dfrac{\hbar}{i}\odv{}{x}\Bigr)\psi,\nonumber        \\
		\avg{\op{p}} & = \int \odif[sep-end=\medspace]{x} \psi^{*}\Bigl(\dfrac{\hbar}{i}\odv{\psi}{x}\Bigr).\label{eq:avg-p}
	\end{align}

	Y,

	\begin{equation*}
		\avg{\op{p}}^{*} = \int \odif[sep-end=\medspace]{x} \psi \Bigl(-\dfrac{\hbar}{i}\odv{\psi^{*}}{x}\Bigr).
	\end{equation*}

	Integrando por partes,

	\begin{align}
		\avg{\op{p}}^{*} & = \psi \Bigl(-\dfrac{\hbar}{i}\psi^{*}\Bigr)_{-\infty}^{\infty} - \int \odif[sep-end=\medspace]{x} \odv{\psi}{x}\Bigl(-\dfrac{\hbar}{i}\psi^{*}\Bigr),\nonumber \\
		\avg{\op{p}}^{*} & = \int \odif[sep-end=\medspace]{x} \psi^{*}\Bigl(\dfrac{\hbar}{i}\odv{\psi}{x}\Bigr). \label{eq:avg-p-star}
	\end{align}

	De las \zcref{eq:avg-p,eq:avg-p-star} tenemos

	\begin{equation*}
		\avg{\op{p}} = \int \odif[sep-end=\medspace]{x} \psi^{*}\Bigl(\dfrac{\hbar}{i}\odv{\psi}{x}\Bigr) = \int \odif[sep-end=\medspace]{x} \psi^{*}\Bigl(\dfrac{\hbar}{i}\odv{\psi}{x}\Bigr) = \avg{\op{p}}^{*}.
	\end{equation*}

	Por lo tanto \(\avg{\op{p}}\) es un operador hermiteano.

\end{problema}
\end{document}
