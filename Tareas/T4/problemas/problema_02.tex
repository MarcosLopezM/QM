\documentclass[../main.tex]{subfiles}

\begin{document}
\begin{problema}[20]
	Si en mecánica usted tiene un producto de la forma:
	\begin{equation*}
		\vect{A}(\vect{r})\cdot \vect{p}
	\end{equation*}

	donde \(\vect{A}\) es el potencial vector y \(\vect{p}\) el
	momentum de una partícula.

	\begin{enumerate}
		\item Muestre que el operador cuántico definido como

		      \begin{equation*}
			      \vect{A}(\vect{r})\cdot \hat{\vect{p}}
		      \end{equation*}

		      no es un operador hermiteano.
		\item Muestre que el operador cuántico definido como

		      \begin{equation*}
			      \dfrac{1}{2}\Bigl(\vect{A}(\vect{r})\cdot \hat{\vect{p}} + \hat{\vect{p}}\cdot \vect{A}(\vect{r})\Bigr)
		      \end{equation*}

		      si es un operador hermiteano.
	\end{enumerate}
\end{problema}
\end{document}
