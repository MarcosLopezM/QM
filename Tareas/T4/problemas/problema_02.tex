\documentclass[../main.tex]{subfiles}

\begin{document}
\begin{problema}[20]
	Si en mecánica usted tiene un producto de la forma:
	\begin{equation*}
		\vect{A}(\vect{r})\cdot \vect{p}
	\end{equation*}

	donde \(\vect{A}\) es el potencial vector y \(\vect{p}\) el
	momentum de una partícula.

	\begin{enumerate}
		\item Muestre que el operador cuántico definido como

		      \begin{equation*}
			      \vect{A}(\vect{r})\cdot \hat{\vect{p}}
		      \end{equation*}

		      no es un operador hermiteano.
		\item Muestre que el operador cuántico definido como

		      \begin{equation*}
			      \dfrac{1}{2}\Bigl(\vect{A}(\vect{r})\cdot \hat{\vect{p}} + \hat{\vect{p}}\cdot \vect{A}(\vect{r})\Bigr)
		      \end{equation*}

		      si es un operador hermiteano.
	\end{enumerate}

	\startsolution

	\section{Inciso (a)}

	Consideremos ahora el operador \(\op{p}\) en tres coordenadas espaciales, i.e.

	\begin{equation*}
		\op{p} = \dfrac{\hbar}{i}\nabla.
	\end{equation*}

	Expresando el operador en notación de índices,

	\begin{equation*}
		\vect{A}(\vect{r}) \cdot \op{\vect{p}} = \dfrac{\hbar}{i}A_{\alpha}\pdv{}{x_{\alpha}}.
	\end{equation*}

	Por lo que,

	\begin{equation}
		\avg{\vect{A}(\vect{r}) \cdot \op{\vect{p}}} = \int \odif[sep-end=\medspace, order = 3]{r} \psi^{*}\Bigl(\dfrac{\hbar}{i}A_{\alpha}\pdv{}{x_{\alpha}}\Bigr)\psi.
	\end{equation}

	Integrando por partes,

	\begin{equation*}
		\avg{\vect{A}(\vect{r}) \cdot \op{\vect{p}}} = \matrixel{\psi}{\vect{A}(\vect{r}) \cdot \op{\vect{p}}}{\psi} = \dfrac{\hbar}{i}\Biggl[\psi^{*}(A_{\alpha}\psi)_{-\infty}^{\infty} - \int \odif[sep-end=\medspace, order = 3]{r} \psi A_{\gamma}\pdv{\psi^{*}}{x_{\gamma}}\Biggr].
	\end{equation*}

	El primer término ``desaparece'', pues sabemos que no hay función de onda en \(\pm \infty\),
	así

	\begin{align*}
		\matrixel{\psi}{\vect{A}(\vect{r}) \cdot \op{\vect{p}}}{\psi}            & = -\int \odif[sep-end=\medspace, order = 3]{r} \psi\Biggl(-A_{\gamma}\dfrac{\hbar}{i}\pdv{\psi}{x_{\gamma}}\Biggr)^{*}, \\
		                                                                         & = -\int \odif[sep-end=\medspace, order = 3]{r} \Biggl(-A_{\gamma}\dfrac{\hbar}{i}\pdv{\psi}{x_{\gamma}}\Biggr)^{*}\psi, \\
		                                                                         & = - \int \odif[order = 3, sep-end=\medspace]{r} (\vect{A}(\vect{r})\cdot \op{\vect{p}})^{*} \psi^{*}\psi,               \\
		\Aboxedmain{\matrixel{\psi}{\vect{A}(\vect{r})\cdot \op{\vect{p}}}{\psi} & \neq \braket{(\vect{A}(\vect{r})\cdot \op{\vect{p}})\psi}{\psi}.}
	\end{align*}

	Por lo tanto, el operador \(\vect{A}(\vect{r})\cdot \op{\vect{p}}\) no es hermiteano.

	\section{Inciso (b)}
\end{problema}
\end{document}
