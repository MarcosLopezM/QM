\documentclass[../main.tex]{subfiles}

\begin{document}
\begin{problema}[20]
	Si en mecánica usted tiene un producto de la forma:
	\begin{equation*}
		\vect{A}(\vect{r})\cdot \vect{p}
	\end{equation*}

	donde \(\vect{A}\) es el potencial vector y \(\vect{p}\) el
	momentum de una partícula.

	\begin{enumerate}
		\item Muestre que el operador cuántico definido como

		      \begin{equation*}
			      \vect{A}(\vect{r})\cdot \hat{\vect{p}}
		      \end{equation*}

		      no es un operador hermiteano.
		\item Muestre que el operador cuántico definido como

		      \begin{equation*}
			      \dfrac{1}{2}\Bigl(\vect{A}(\vect{r})\cdot \hat{\vect{p}} + \hat{\vect{p}}\cdot \vect{A}(\vect{r})\Bigr)
		      \end{equation*}

		      si es un operador hermiteano.
	\end{enumerate}

	\startsolution

	\section{Inciso (a)}

	Consideremos ahora el operador \(\op{p}\) en tres coordenadas espaciales, i.e.

	\begin{equation*}
		\op{p} = \dfrac{\hbar}{i}\nabla.
	\end{equation*}

	Expresando el operador en notación de índices,

	\begin{equation*}
		\vect{A}(\vect{r}) \cdot \op{\vect{p}} = \dfrac{\hbar}{i}A_{\alpha}\pdv{}{x_{\alpha}}.
	\end{equation*}

	Por lo que,

	\begin{equation*}
		\avg{\vect{A}(\vect{r}) \cdot \op{\vect{p}}} = \int \odif[sep-end=\medspace, order = 3]{r} \psi^{*}\Bigl(\dfrac{\hbar}{i}A_{\alpha}\pdv{}{x_{\alpha}}\Bigr)\psi.
	\end{equation*}

	Integrando por partes,

	\begin{equation*}
		\avg{\vect{A}(\vect{r}) \cdot \op{\vect{p}}} = \matrixel{\psi}{\vect{A}(\vect{r}) \cdot \op{\vect{p}}}{\psi} = \dfrac{\hbar}{i}\Biggl[\psi^{*}(A_{\alpha}\psi)_{-\infty}^{\infty} - \int \odif[sep-end=\medspace, order = 3]{r} \psi A_{\gamma}\pdv{\psi^{*}}{x_{\gamma}}\Biggr].
	\end{equation*}

	El primer término ``desaparece'', pues sabemos que no hay función de onda en \(\pm \infty\),
	así

	\begin{align*}
		\matrixel{\psi}{\vect{A}(\vect{r}) \cdot \op{\vect{p}}}{\psi}            & = \int \odif[sep-end=\medspace, order = 3]{r} \psi\Biggl(-\dfrac{\hbar}{i}\odv{}{x_{\gamma}}A_{\gamma}\psi\Biggr)^{*}, \\
		                                                                         & = \int \odif[sep-end=\medspace, order = 3]{r} \Biggl(-\dfrac{\hbar}{i}\odv{}{x_{\gamma}}A_{\gamma}\psi\Biggr)^{*}\psi, \\
		                                                                         & =  \int \odif[order = 3, sep-end=\medspace]{r} (\op{\vect{p}}\cdot \vect{A}(\vect{r}))^{*} \psi^{*}\psi,               \\
		\Aboxedmain{\matrixel{\psi}{\vect{A}(\vect{r})\cdot \op{\vect{p}}}{\psi} & \neq \braket{(\op{\vect{p}}\cdot \vect{A}(\vect{r}))\psi}{\psi}.}
	\end{align*}

	Por lo tanto, el operador \(\vect{A}(\vect{r})\cdot \op{\vect{p}}\) no es hermiteano.

	\section{Inciso (b)}

	Calculamos

	\begin{align}
		\avg`*{\dfrac{1}{2}\Bigl[\vect{A}(\vect{r})\cdot \op{\vect{p}} + \op{\vect{p}}\cdot \vect{A}(\vect{r})\Bigr]} & = \int \odif[order = 3, sep-end=\medspace]{r} \psi^{*}\dfrac{1}{2}\Bigl[A_{\alpha}\dfrac{\hbar}{i}\pdv{}{x_{\alpha}} + \dfrac{\hbar}{i}\pdv{}{x_{\alpha}}A_{\alpha}\Bigr]\psi,\nonumber                                                                                                    \\
		\avg`*{\dfrac{1}{2}\Bigl[\vect{A}(\vect{r})\cdot \op{\vect{p}} + \op{\vect{p}}\cdot \vect{A}(\vect{r})\Bigr]} & = \dfrac{1}{2}\unbr{ \int \odif[sep-end=\medspace, order = 3]{r} \psi^{*}A_{\alpha}\dfrac{\hbar}{i}\odv{\psi}{x_{\alpha}}}_{(A)} + \dfrac{1}{2}\unbr{ \int \odif[sep-end=\medspace, order = 3]{r} \psi^{*}\dfrac{\hbar}{i}\pdv{(A_{\beta}\psi)}{x_{\beta}}}_{(B)}.\label{eq:avg-normal-op}
	\end{align}

	Integrando por partes (A),

	\begin{align}
		\int \odif[sep-end=\medspace, order = 3]{r} \psi^{*}A_{\alpha}\dfrac{\hbar}{i}\odv{\psi}{x_{\alpha}} & = \psi^{*}\Bigl(A_{\alpha}\dfrac{\hbar}{i}\psi\Bigr)\Bigr\rvert_{-\infty}^{\infty} - \int \odif[order = 3, sep-end=\medspace]{r} \psi A_{\alpha}\dfrac{\hbar}{i}\pdv{\psi^{*}}{x_{\alpha}},\nonumber \\
		                                                                                                     & = \int \odif[sep-end=\medspace, order = 3]{r} \Bigl(-\dfrac{\hbar}{i}\pdv{}{x_{\alpha}}A_{\alpha}\psi\Bigr)^{*}\psi,\nonumber                                                                        \\
		                                                                                                     & = \braket`*{(\op{\vect{p}}\cdot \vect{A}(\vect{r}))\psi}{\psi}.\label{eq:first-term}
	\end{align}

	Integrando por partes (B),

	\begin{align}
		\int \odif[sep-end=\medspace, order = 3]{r} \psi^{*}\dfrac{\hbar}{i}\pdv{(A_{\beta}\psi)}{x_{\beta}} & = \psi^{*}\Bigl(\dfrac{\hbar}{i}A_{\alpha}\psi\Bigr)\Bigr\rvert_{-\infty}^{\infty} - \int \odif[sep-end=\medspace, order = 3]{r} \psi A_{\alpha}\dfrac{\hbar}{i}\pdv{\psi^{*}}{x_{\alpha}},\nonumber \\
		                                                                                                     & = \int \odif[sep-end=\medspace, order = 3]{r} \Bigl(-A_{\alpha}\dfrac{\hbar}{i}\pdv{}{x_{\alpha}\psi}\Bigr)^{*}\psi,\nonumber                                                                        \\
		                                                                                                     & = \braket`*{(\vect{A}(\vect{r})\cdot \op{\vect{p}})\psi}{\psi}.\label{eq:second-term}
	\end{align}

	Sustituyendo las \zcref{eq:first-term,eq:second-term} en la \zcref{eq:avg-normal-op},

	\begin{align*}
		\dfrac{1}{2}\matrixel{\psi}{\vect{A}(\vect{r})\cdot \op{\vect{p}}}{\psi} + \dfrac{1}{2}\matrixel{\psi}{\op{\vect{p}}\cdot \vect{A}(\vect{r})}{\psi} & = \dfrac{1}{2}\braket`*{(\op{\vect{p}}\cdot \vect{A}(\vect{r}))\psi}{\psi} + \dfrac{1}{2}\braket`*{(\vect{A}(\vect{r})\cdot \op{\vect{p}})\psi}{\psi}, \\
		\dfrac{1}{2}\matrixel{\psi}{\vect{A}(\vect{r})\cdot \op{\vect{p}}}{\psi} + \dfrac{1}{2}\matrixel{\psi}{\op{\vect{p}}\cdot \vect{A}(\vect{r})}{\psi} & = \dfrac{1}{2}\braket`*{(\vect{A}(\vect{r})\cdot \op{\vect{p}})\psi}{\psi} + \dfrac{1}{2}\braket`*{(\op{\vect{p}}\cdot \vect{A}(\vect{r}))\psi}{\psi}.
	\end{align*}

	Por lo tanto, el operador es hermiteano.
\end{problema}
\end{document}
