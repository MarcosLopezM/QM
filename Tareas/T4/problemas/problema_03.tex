\documentclass[../main.tex]{subfiles}

\begin{document}
\begin{problema}
	En clase vimos que al trabajar un oscilador armónico cuántico, la ecuación de
	Schrödinger independiente del tiempo puede reescribirse como

	\begin{equation*}
		\op{H}\varphi(x) = E\varphi(x)
	\end{equation*}

	donde

	\begin{equation*}
		\op{H} = \hbar\omega \Bigl(\hc{a}\op{a} + \dfrac{1}{2}\Bigr),
	\end{equation*}

	con \(\hc{a} \equiv \odv{}{x} + u\) y \( \op{a} \equiv \odv{}{u} + u\), los
	operadores de ascenso y descenso de energía, respectivamente. Muestre que

	\begin{enumerate}
		\item Los operadores de ascenso y descenso no son operadores hermiteanos.
		\item Que la acción del operador hacia la derecha
		      \begin{equation*}
			      \op{a} \ket{\varphi},
		      \end{equation*}

		      baja en un cuanto de energía, es decir

		      \begin{equation*}
			      \op{H}\op{a}\ket{\varphi} = \hbar\omega(\mathcal{E} - 1)\ket{\varphi},
		      \end{equation*}

		      y que al actuar hacia la izquierda el mismo operador, éste aumenta en un
		      cuanto la energía, es decir

		      \begin{equation}
			      \bigl(\bra{\varphi}\op{a}\bigr)\op{H} = \bigl(\bra{\varphi}\op{a}\bigr)\hbar\omega(\mathcal{E} + 1).
		      \end{equation}
	\end{enumerate}
\end{problema}
\end{document}
