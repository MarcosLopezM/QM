\documentclass[../main.tex]{subfiles}

\begin{document}
\begin{problema}[20]
	En clase vimos que en la mecánica cuántica, cuando trabajamos con un
	paquete de onda que describe una partícula libre de masa \(m\), existe
	una ecuación de continuidad de la forma

	\begin{equation*}
		\pdv*{\rho}{t} + \nabla \cdot{\vect{j}} = 0
	\end{equation*}

	donde

	\begin{align*}
		\rho     & = \psi^{*}\psi ,                                                                 \\
		\vect{j} & = \dfrac{\hbar}{2mi}\Bigl[\psi^{*}(\nabla{\psi}) - (\nabla{\psi^{*}})\psi\Bigr].
	\end{align*}

	La cual físicamente nos indica que los cambios temporales en
	la densidad de probabilidad \(\rho\) se relacionan a divergencias
	en la densidad de corriente de probabilidad \(\vect{j}\).

	La pregunta es: ¿cómo se modifica esta ecuación de continuidad
	si la partícula tiene carga \(q\), masa \(m\), y se encuentra en
	presencia de un campo electromagnético?
\end{problema}
\end{document}
