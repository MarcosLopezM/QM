\documentclass[../main.tex]{subfiles}

\begin{document}
\begin{problema}[20]
	En clase vimos que en la mecánica cuántica, cuando trabajamos con un
	paquete de onda que describe una partícula libre de masa \(m\), existe
	una ecuación de continuidad de la forma

	\begin{equation*}
		\pdv*{\rho}{t} + \nabla \cdot{\vect{j}} = 0
	\end{equation*}

	donde

	\begin{align*}
		\rho     & = \psi^{*}\psi ,                                                                 \\
		\vect{j} & = \dfrac{\hbar}{2mi}\Bigl[\psi^{*}(\nabla{\psi}) - (\nabla{\psi^{*}})\psi\Bigr].
	\end{align*}

	La cual físicamente nos indica que los cambios temporales en
	la densidad de probabilidad \(\rho\) se relacionan a divergencias
	en la densidad de corriente de probabilidad \(\vect{j}\).

	La pregunta es: ¿cómo se modifica esta ecuación de continuidad
	si la partícula tiene carga \(q\), masa \(m\), y se encuentra en
	presencia de un campo electromagnético?
\end{problema}

\startsolution

La ecuación de Schrödinger para una partícula con carga \(q\),
masa \(m\) y que es encuentra en presencia de un campo electromagnético
es:

\begin{equation}
	i\hbar \dot{\psi} = \dfrac{1}{2m}\Biggl[\unbr{ \Bigl(\tfrac{\hbar}{i}\nabla - \tfrac{q}{c}\vect{A}\Bigr)^{2} }_{(A)} + q\phi\Biggr]\psi.
	\label{eq:Schr-electromagnetic}
\end{equation}

Desarrollamos (A) de la expresión anterior,

\begin{align*}
	\Bigl(\dfrac{\hbar}{i}\nabla - \dfrac{q}{c}\vect{A}\Bigr)\cdot\Bigl(\dfrac{\hbar}{i}\nabla - \dfrac{q}{c}\vect{A}\Bigr) & =
	-\hbar^{2}\nabla \cdot{\nabla} - \dfrac{\hbar q}{ic}\nabla \cdot{\vect{A}} - \dfrac{\hbar q}{ic}\vect{A}\cdot\nabla + \dfrac{q^{2}}{c^{2}}\vect{A}\cdot \vect{A},                                                                                              \\
	                                                                                                                        & = -\hbar^{2}\nabla^{2} - \dfrac{\hbar q}{ic}(\nabla \cdot{\vect{A}} - \vect{A}\cdot\nabla) + \dfrac{q^{2}}{c^{2}}\abs{\vect{A}}^{2}.
\end{align*}

Sustituyendo en la \zcref{eq:Schr-electromagnetic} y desarrollando,

\begin{equation*}
	i\hbar \dot{\psi} = \dfrac{1}{2m}\Biggl[-\hbar^{2}\nabla^{2}{\psi} - \dfrac{\hbar q}{ic}\unbr{ \Bigl((\nabla \cdot{\vect{A}})\psi - \vect{A}\cdot\nabla \cdot{\psi}\Bigr) }_{(B)} + \dfrac{q^{2}}{c^{2}}\abs{\vect{A}}\psi + q\phi\psi\Biggr].
\end{equation*}

\pagebreak
Notemos que podemos escribir (B) como \(\nabla \cdot{(\vect{A}\psi)}\),

\begin{equation}
	i\hbar \dot{\psi} = \dfrac{1}{2m}\Biggl[-\hbar^{2}\nabla^{2}{\psi} - \dfrac{\hbar q}{ic}\nabla \cdot{(\vect{A}\psi)} + \dfrac{q^{2}}{c^{2}}\abs{\vect{A}}\psi + q\phi\psi\Biggr].
	\label{eq:Schrodinger-computed}
\end{equation}

Calculamos el conjugado de la \zcref{eq:Schrodinger-computed},

\begin{equation}
	-i\hbar \dot{\psi}^{*} = \dfrac{1}{2m}\Biggl[-\hbar^{2}\nabla^{2}{\psi^{*}} + \dfrac{\hbar q}{ic}\nabla \cdot{(\vect{A}\psi^{*})} + \dfrac{q^{2}}{c^{2}}\abs{\vect{A}}^{2}\psi^{*} + q\phi\psi^{*}\Biggr].
	\label{eq:Schrodinger-conjugate}
\end{equation}

Ahora, multiplicamos la \zcref{eq:Schrodinger-computed} por \(\psi^{*}\),

\begin{equation}
	i\hbar\psi^{*}\dot{\psi} = \dfrac{1}{2m}\Biggl[-h^{2}\psi^{*}\nabla^{2}{\psi} - \dfrac{\hbar q}{ic}\psi^{*}\nabla \cdot{(\vect{A}\psi)} + \dfrac{q^{2}}{c^{2}}\abs{\vect{A}}^{2}\psi^{*}\psi + q\phi\psi^{*}\psi\Biggr].
	\label{eq:Schrodinger-computed-product}
\end{equation}

Y, la \zcref{eq:Schrodinger-conjugate} por \(\psi\),

\begin{equation}
	-i\hbar \dot{\psi}^{*}\psi = \dfrac{1}{2m}\Biggl[-\hbar^{2}(\nabla^{2}{\psi^{*}})\psi + \dfrac{\hbar q}{ic}\nabla \cdot{(\vect{A}\psi^{*})}\psi + \dfrac{q^{2}}{c^{2}}\abs{\vect{A}}\psi^{*}\psi + q\phi\psi^{*}\psi\Biggr].
	\label{eq:Schrodinger-conjugate-product}
\end{equation}

Restando la \zcref{eq:Schrodinger-conjugate-product} de la \zcref{eq:Schrodinger-computed-product},

\begin{equation*}
	i\hbar(\psi^{*}\psi + \dot{\psi}^{*}\psi) = \dfrac{1}{2m}\Biggl[-\hbar^{2}\unbr{\Bigl(\psi^{*}\nabla^{2}{\psi} - (\nabla^{2}{\psi^{*}})\psi\Bigr)}_{(C)} - \dfrac{\hbar q}{ic}\Bigl(\psi^{*}\nabla \cdot{(\vect{A}\psi)} + (\nabla \cdot{(\vect{A}\psi^{*})})\psi\Bigr) \Biggr].
\end{equation*}

Por el problema anterior sabemos que podemos escribir (C) como divergencias,
el primer término como

\begin{align*}
	\nabla \cdot{(\psi^{*}\nabla{\psi})} & = (\nabla{\psi^{*}})\cdot(\nabla{\psi}) + \psi^{*}\nabla^{2}{\psi},             \\
	\psi^{*}\nabla^{2}{\psi}             & = \nabla \cdot{(\psi^{*}\nabla{\psi})} - (\nabla{\psi^{*}})\cdot(\nabla{\psi}).
\end{align*}

y el segundo,

\begin{align*}
	\nabla \cdot{(\nabla{\psi^{*}})\psi} & = (\nabla^{2}{\psi^{*}})\psi + (\nabla{\psi^{*}})\cdot(\nabla{\psi}),           \\
	(\nabla^{2}{\psi^{*}})\psi           & = \nabla \cdot{(\nabla{\psi^{*}})\psi} - (\nabla{\psi^{*}})\cdot(\nabla{\psi}).
\end{align*}

Entonces,

\begin{equation*}
	i\hbar(\psi^{*}\psi + \dot{\psi}^{*}\psi) = \dfrac{1}{2m}\Biggl[-\hbar^{2}\Bigl(\nabla \cdot{(\psi^{*}\nabla{\psi})} - \nabla \cdot{((\nabla{\psi^{*}})\psi)}\Bigr) - \dfrac{\hbar q}{ic}\unbr{\Bigl(\psi^{*}(\nabla \cdot{(\vect{A}\psi)}) + (\nabla \cdot{(\vect{A}\psi^{*})})\psi\Bigr)}_{(D)}\Biggr]
\end{equation*}

Desarrollando (D) por componentes:

\begin{align*}
	\psi^{*}\odv{(A_{i}\psi)}{x_{i}} + \odv{(A_{i}\psi^{*})}{x_{i}}\psi & = \psi^{*}\Biggl[\odv{A_{i}}{x_{i}}\psi + A_{i}\odv{\psi}{x_{i}}\Biggr] + \Biggl[\odv{A_{i}}{x_{i}}\psi^{*} + A_{i}\odv{\psi^{*}}{x_{i}}\Biggr]\psi,      \\
	                                                                    & = (\nabla \cdot{\vect{A}})\psi^{*}\psi + \psi^{*}\vect{A}\cdot \nabla{\psi} + (\nabla \cdot{\vect{A}})\psi^{*}\psi + \vect{A}\cdot(\nabla{\psi^{*}\psi}), \\
	                                                                    & = 2(\nabla \cdot{\vect{A}})\psi^{*}\psi + \nabla \cdot{(\vect{A}\psi^{*}\psi)}.
\end{align*}

Eligiendo \(\nabla \cdot{\vect{A}} = 0\), tenemos que:

\begin{equation*}
	\pdv{(\psi^{*}\psi)}{t} = -\nabla \cdot{\Biggl[\dfrac{\hbar}{2mi}(\psi^{*}\nabla{\psi} - (\nabla{\psi^{*}})\psi) + \dfrac{q}{2mci}\vect{A}\psi^{*}\psi\Biggr]},
\end{equation*}

donde

\begin{empheq}[box = \mainresult]{align*}
	\rho &= \psi^{*}\psi,\\
	\vect{j} &= \dfrac{\hbar}{2mi}(\psi^{*}\nabla{\psi} - (\nabla{\psi^{*}})\psi) + \dfrac{q}{2mci}\vect{A}\psi^{*}\psi.
\end{empheq}

Por lo tanto, la ecuación de continuidad para una partícula de carga \(q\), masa
\(m\) y que se encuentra en presencia de un campo electromagnético es

\begin{empheq}[box = \mainresult]{equation*}
	\pdv{\rho}{t} + \nabla \cdot{\vect{j}} = 0.
\end{empheq}
\end{document}
