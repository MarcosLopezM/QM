\documentclass[../main.tex]{subfiles}

\begin{document}
\begin{problema}[10]
	En clase vimos que al cuantificar el cambio en el tiempo de la energía
	cinética, teníamos

	\begin{equation*}
		\odv{E_{\text{pot}}}{t} = \dfrac{\hbar^{2}}{2m}\Biggl[\int \odif[order=3,sep-end=\medspace]{x} \dot{\psi}^{*}(\nabla^{2}{\psi}) + (\nabla^{2}{\psi^{*}})\dot{\psi}\Biggr].
	\end{equation*}

	El problema consiste en que muestren el cambio en el tiempo de la
	ecuación anterior se puede escribir como

	\begin{equation*}
		\odv{E_{\text{pot}}}{t} = \odv*{\Biggl[\dfrac{\hbar^{2}}{2m}\int \odif[sep-end=\medspace, order=3]{x} \dot{\psi}^{*}\nabla^{2}{\psi}\Biggr]}{t}.
	\end{equation*}
\end{problema}
\end{document}

