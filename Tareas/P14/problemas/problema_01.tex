\documentclass[../main.tex]{subfiles}

\begin{document}
\begin{problema}
	Reducir las expresiones obtenidas a partir de las condiciones de frontera para el pozo de potencial finito a las ecuaciones

	\begin{align*}
		k_{2} \cot \Bigl(\dfrac{k_{2}a}{2}\Bigr) & = -k_{1}, \\
		k_{2}\tan \Bigl(\dfrac{k_{2}a}{2}\Bigr)  & = k_{1}.
	\end{align*}

	\startsolution

	Consideramos el problema del pozo de potencial finito para el caso
	en el que \(E < V_{0}\), donde el potencial está dado por

	\begin{equation*}
		V(x) =
		\begin{dcases}
			V_{0} & \text{si } x < -\tfrac{a}{2},                      \\
			0     & \text{si } -\tfrac{a}{2} \leq x \leq \tfrac{a}{2}, \\
			V_{0} & \text{si } x > \tfrac{a}{2}.
		\end{dcases}
	\end{equation*}

	Partiendo de la ecuación de Schrödinger independiente del tiempo, tenemos
	que las soluciones en las tres regiones son:

	\begin{itemize}
		\item Región I (\(x < -\tfrac{a}{2}\))

		      \begin{equation*}
			      \psi_{I}(x) = A \mathrm{e}^{k_{1} x}
		      \end{equation*}

		      El término creciente se descarta, ya que cuando \(x \to -\infty\), este
		      diverge.

		\item Región II

		      Para esta región, \(-\tfrac{a}{2} < x < \tfrac{a}{2}\), la solución es

		      \begin{equation*}
			      \psi_{II}(x) = B \sin\bigl(k_{2}x\bigr) + C\cos\bigl(k_{2}x\bigr).
		      \end{equation*}

		\item Región III

		      Un análisis análogo al de la región I nos lleva a la solución

		      \begin{equation*}
			      \psi_{III}(x) = D \mathrm{e}^{-k_{1}x}.
		      \end{equation*}
	\end{itemize}

	Para encontrar la solución completa, debemos imponer las condiciones de frontera

	\begin{align*}
		\psi_{I}\bigl(-\tfrac{a}{2}\bigr) & = \psi_{II}\bigl(-\tfrac{a}{2}\bigr), \\
		\psi_{II}\bigl(\tfrac{a}{2}\bigr) & = \psi_{III}\bigl(\tfrac{a}{2}\bigr).
	\end{align*}

	Y, además, tanto la función de onda como su derivada deben ser continuas en la vecindad
	de la frontera, es decir,

	\begin{align*}
		\op{p}\psi_{I}\bigl(-\tfrac{a}{2}\bigr) & = \op{p}\psi_{II}\bigl(-\tfrac{a}{2}\bigr), \\
		\op{p}\psi_{II}\bigl(\tfrac{a}{2}\bigr) & = \op{p}\psi_{III}\bigl(\tfrac{a}{2}\bigr).
	\end{align*}

	Por lo tanto, tenemos que el sistema de ecuaciones es

	\begin{align}
		A \mathrm{e}^{-k_{1}a/2}      & = C \cos\Bigl(-k_{2}\dfrac{a}{2}\Bigr) + B \sin\Bigl(-k_{2}\dfrac{a}{2}\Bigr),\label{eq:psi_neg}         \\
		D \mathrm{e}^{-k_{1}a/2}      & = C \cos\Bigl(k_{2}\dfrac{a}{2}\Bigr) + B \sin\Bigl(k_{2}\dfrac{a}{2}\Bigr),\label{eq:psi_pos}           \\
		Ak_{1}\mathrm{e}^{-k_{1}a/2}  & = -Ck_{2}\sin\Bigl(-k_{2}\dfrac{a}{2}\Bigr) + Bk_{2}\cos\Bigl(-k_{2}\dfrac{a}{2}\Bigr),\label{eq:dv_neg} \\
		-Dk_{1}\mathrm{e}^{-k_{1}a/2} & = -Ck_{2}\sin\Bigl(k_{2}\dfrac{a}{2}\Bigr) + Bk_{2}\cos\Bigl(k_{2}\dfrac{a}{2}\Bigr)\label{eq:dv_pos}.
	\end{align}

	Para reducir este sistema a las ecuaciones que queremos, dado que el
	potencial es simétrico, las soluciones puede clasificarse según su
	paridad. Primero, para las soluciones pares, tenemos que \(B = 0\),
	entonces las \zcref{eq:psi_neg,eq:dv_neg} quedan como

	\begin{align}
		A \mathrm{e}^{-k_{1}a/2}     & = C \cos\Bigl(k_{2}\dfrac{a}{2}\Bigr),\label{eq:even_psi_neg}    \\
		Ak_{1}\mathrm{e}^{-k_{1}a/2} & = Ck_{2}\sin\Bigl(k_{2}\dfrac{a}{2}\Bigr).\label{eq:even_dv_neg}
	\end{align}

	Calculando el cociente entre las \zcref{eq:even_dv_neg, eq:even_psi_neg},

	\begin{align*}
		k_{2}\dfrac{\sin\Bigl(k_{2}\dfrac{a}{2}\Bigr)}{\cos\Bigl(k_{2}\dfrac{a}{2}\Bigr)} & = k_{1},  \\
		\Aboxedmain{k_{2}\tan\Bigl(\dfrac{k_{2}a}{2}\Bigr)                                & = k_{1}.}
	\end{align*}

	Mientras que para las soluciones impares, \(C = 0\),
	tal que las \zcref{eq:psi_pos,eq:dv_pos} quedan como:

	\begin{align*}
		D \mathrm{e}^{-k_{1}a/2}      & = B \sin\Bigl(k_{2}\dfrac{a}{2}\Bigr),     \\
		-Dk_{1}\mathrm{e}^{-k_{1}a/2} & = Bk_{2}\cos\Bigl(k_{2}\dfrac{a}{2}\Bigr).
	\end{align*}

	Así, el cociente entre estas dos ecuaciones es:

	\begin{align*}
		k_{2}\dfrac{\cos\Bigl(k_{2}\dfrac{a}{2}\Bigr)}{\sin\Bigl(k_{2}\dfrac{a}{2}\Bigr)} & = -k_{1},  \\
		\Aboxedmain{k_{2} \cot\Bigl(\dfrac{k_{2}a}{2}\Bigr)                               & = -k_{1}.}
	\end{align*}
\end{problema}
\end{document}
