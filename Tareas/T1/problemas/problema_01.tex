%! TeX program = lualatex
\documentclass[../main.tex]{subfiles}

\begin{document}
\begin{problema}[10]
	A partir de las constante fundamentales \(c, h, G, \kappa_{B}\),
	(la velocidad del vacío, la constante de Planck, la constante de Gravitación universal
	y la constante de Boltzmann, respectivamente),
	mediante un análisis dimensional, construya las
	\textbf{unidades de Planck} asociadas a una escala de longitud,
	una escala de masa, una escala de tiempo y una escala de temperatura.

	\startsolution

	Para construir las unidades de Planck en términos de las constantes
	fundamentales, recordamos las dimensiones de cada una de estas,

	\begin{align*}
		[c]          & = [LT^{-1}],            \\
		[h]          & = [ML^{2}T^{-1}],       \\
		[G]          & = [L^{3}T^{-2}M^{-1}],  \\
		[\kappa_{B}] & = [ML^{2}T^{-2}K^{-1}].
	\end{align*}

	Comencemos construyendo la longitud de Planck, que denotaremos por
	\(\ell_{P}\), jugando con las constantes de tal forma que la dimensión
	sea \(L\).

	\begin{align}
		\dfrac{h}{c}            & = \dfrac{\tfrac{ML^{2}}{T}}{\tfrac{L}{T}},\nonumber \\
		\Aboxedsec{\dfrac{h}{c} & = ML.}\label{eq:hDividedbyC}
	\end{align}

	La expresión anterior se acerca demasiado al resultado, excepto por un factor de \(M\),
	por lo que ahora intentamos "deshacernos" de este multiplicando por \(G\),

	\begin{equation}
		\biggl[\dfrac{hG}{c}\biggr] = (ML)\dfrac{L^{3}}{T^{2}M} = \dfrac{L^{4}}{T^{2}}.
		\label{eq:hGDividedbyC}
	\end{equation}

	Recordamos por la \zcref{eq:hDividedbyC} que al dividir por \(c\)
	nos deshicimos de las dimensiones de \(L\) y \(T\), por lo que podemos
	repetir este paso, ahora por \(c^{2}\),

	\begin{equation}
		\biggl[\dfrac{hG}{c^{3}}\biggr] = \dfrac{L^{4}}{T^{2}}\dfrac{T^{2}}{L^{2}} = L^{2}.
		\label{eq:hGDividedbyC3}
	\end{equation}

	Basta sacar la raíz cuadrada de la expresión anterior para obtener \(\ell_{P}\),

	\begin{align*}
		\Biggl[\sqrt{\dfrac{hG}{c^{3}}}\Biggr] & = \sqrt{L^{2}} = \abs{L} = L, \\
		\Aboxedmain{[\ell_{P}]                 & = [L]
		.
		}
	\end{align*}

	Para la masa de Planck, \(m_{P}\), nos fijamos que la \zcref{eq:hDividedbyC}
	tiene dimensiones que nos interesan, ahora intentemos multiplicándolas, i.e.

	\begin{equation}
		[hc] = \bigl[\tfrac{ML^{2}}{T}\bigr]\bigl[\tfrac{L}{T}\bigr] = \dfrac{ML^{3}}{T^{2}}.
		\label{eq:hTimesC}
	\end{equation}

	Entonces, si dividimos la \zcref{eq:hTimesC} por
	\(G\) tenemos

	\begin{align*}
		\Bigl[\dfrac{gc}{G}\Bigr] & = \dfrac{[hc]}{[G]} = \dfrac{\biggl[\tfrac{ML^{3}}{T^{2}}\biggr]}{\biggl[\tfrac{L^{3}}{T^{2}M}\biggr]}
		= \Biggl[\dfrac{M^{2}L^{3}T^{2}}{T^{2}L^{3}}\Biggr],                                                                               \\
		\Bigl[\dfrac{gc}{G}\Bigr] & = [M^{2}].
	\end{align*}

	El paso final es análogo al que se hizo para \(\ell_{P}\), por lo tanto,

	\begin{empheq}[box=\mainresult]{equation*}
		[m_{P}] = \Biggl[\sqrt{\dfrac{hc}{G}}\Biggr] = [M].
	\end{empheq}

	Si inspeccionamos alguno de los resultados anteriores notamos que
	ninguno parece tener ``dimensiones extras'' en \(T\),
	entonces analicemos el siguiente producto

	\begin{equation}
		[Gh] = \biggl[\tfrac{L^{3}}{T^{2}M}\biggr]\biggl[\tfrac{ML^{2}}{T}\biggr]
		= \biggl[\tfrac{L^{5}}{T^{3}}\biggr].
		\label{eq:GTimesH}
	\end{equation}

	Vemos que este producto tiene únicamente dimensiones de \(L\) y \(T\),
	por lo que el paso razonable sería dividir por \(c\), en particular
	\(c^{5}\) por el término \([L^{5}]\),

	\begin{equation*}
		\Biggl[\dfrac{Gh}{c^{5}}\Biggr] = \dfrac{\biggl[\tfrac{L^{5}}{T^{3}}\biggr]}{\biggl[\tfrac{T^{5}L^{5}}{T^{3}L^{5}}\biggr]}
		= [T^{2}].
	\end{equation*}

	Finalmente, el tiempo de Planck, \(t_{P}\), es

	\begin{empheq}[box=\mainresult]{equation*}
		[t_{P}] = \Biggl[\sqrt{\dfrac{Gh}{c^{5}}}\Biggr] = [T].
	\end{empheq}

	Para la temperatura de Planck, \(T_{P}\), primero analizamos lo que sucede con
	\(\bigl[\tfrac{1}{\kappa_{B}}\bigr]\). Sin embargo, notamos que es de mayor
	utilidad tener la dimensión deseada para la unidad de Planck al cuadrado, por
	lo que analizamos \(\bigl[\tfrac{1}{{\kappa_{B}}^{2}}\bigr]\),

	\begin{equation}
		\biggl[\dfrac{1}{{\kappa_{B}}^{2}}\biggr] =
		\dfrac{1}{\Biggl[\biggl(\tfrac{ML^{2}}{T^{2}K}\biggr)^{2}\Biggr]} =
		\Biggl[\dfrac{T^{4}K^{2}}{M^{2}L^{4}}\Biggr].
		\label{eq:OneOverKb2}
	\end{equation}

	Veamos qué pasa si multiplicamos la \zcref{eq:OneOverKb2} por
	\zcref{eq:hTimesC},

	\begin{equation*}
		\Biggl[\dfrac{hc}{{\kappa_{B}}^{2}}\Biggr] =
		\biggl[\tfrac{T^{4}K^{2}}{M^{2}L^{4}}\biggr] =
		\biggl[\tfrac{ML^{3}}{T^{2}}\biggr] =
		\biggl[\tfrac{T^{2}K^{2}}{M}\biggr].
	\end{equation*}

	Si dividimos por \(G\) nos desharemos de la dimensión, i.e.

	\begin{equation*}
		\Biggl[\dfrac{hc}{G{\kappa^{B}}^{2}}\Biggr] =
		\dfrac{\biggl[\tfrac{T^{2}K^{2}}{ML}\biggr]}{\biggl[\tfrac{L^{3}}{T^{2}M}\biggr]} =
		\biggl[\tfrac{T^{4}MK^{2}}{ML^{4}}\biggr] =
		\biggl[\tfrac{T^{4}K^{2}}{L^{4}}\biggr].
	\end{equation*}

	Finalmente podemos multiplicar por \(c^{4}\) pues las dimensiones presentes
	en la expresión anterior son de la forma \(\bigl[\tfrac{1}{c}\bigr] = \bigl[\tfrac{T}{L}\bigr]\). Entonces,

	\begin{equation*}
		\Biggl[\dfrac{hc^{5}}{G{\kappa_{B}}^{2}}\Biggr] =
		\biggl[\tfrac{T^{4}K^{2}}{L^{4}}\biggr]\biggl[\tfrac{L^{4}}{T^{4}}\biggr] =
		[K^{2}].
	\end{equation*}

	Por lo tanto,

	\begin{empheq}[box=\mainresult]{equation*}
		T_{P} = \Biggl[\sqrt{\dfrac{hc^{5}}{G{\kappa_{B}}^{2}}}\Biggr] = [K].
	\end{empheq}
\end{problema}
\end{document}
