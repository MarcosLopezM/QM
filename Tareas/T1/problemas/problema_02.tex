%! TeX program = lualatex
\documentclass[../main.tex]{subfiles}

\begin{document}
\setproblem{1}
\begin{problema}[10]
	Muestre que
	\begin{equation*}
		\int_{-\infty}^{\infty} \e^{-ax^{2}} \odif{x} = \sqrt{\dfrac{\pi}{a}}.
	\end{equation*}

	\startsolution

	A priori sabemos que la siguiente integral es igual a un número \(I\),

	\begin{equation*}
		\int_{-\infty}^{\infty} \e^{-x^{2}} \odif{x} = I.
	\end{equation*}

	Por lo que podemos hacer lo siguiente,

	\begin{equation*}
		I^{2} = I \cdot I = \biggl(\int_{-\infty}^{\infty} \mathrm{e}^{-ax^{2}}\odif{x}\biggr)\biggl(\int_{-\infty}^{\infty} \mathrm{e}^{-ax^{2}}\odif{x}\biggr).
	\end{equation*}

	Notemos que \(x\) es un índice mudo, por lo que podemos escribir
	la expresión anterior como

	\begin{equation*}
		I^{2} = I \cdot I = \biggl(\int_{-\infty}^{\infty} \mathrm{e}^{-ax^{2}}\odif{x}\biggr)\biggl(\int_{-\infty}^{\infty} \mathrm{e}^{-ay^{2}}\odif{y}\biggr).
	\end{equation*}

	Por Fubini, la expresión anterior la podemos escribir como

	\begin{equation*}
		I^{2} = \int_{-\infty}^{\infty}\int_{-\infty}^{\infty}\odif{x}\odif{y}\;\mathrm{e}^{-a(x^{2} + y^{2})}.
	\end{equation*}

	Ahora hacemos un cambio a coordenadas polares con \(r^{2} = x^{2} + y^{2}\) y \(\odif{x}\odif{y} = r \odif{r}\odif{\theta}\). Mientras que los límites de integración
	son \(r\in[0, \infty)\) y \(\theta \in [0, 2\pi)\).

	\begin{equation*}
		I^{2} = \int_{0}^{n}\odif{r}\int_{0}^{2\pi}r \odif{\theta} \mathrm{e}^{-ar^{2}} =
		\int_{0}^{\infty}2\pi r \odif{r} \mathrm{e}^{-ar^{2}}.
	\end{equation*}

	Pero \(\odif{r^{2}} = 2r \odif{r}\),

	\begin{equation*}
		I^{2} = \int_{0}^{\infty}\pi \odif{r^{2}}\mathrm{e}^{-ar^{2}}.
	\end{equation*}

	Nuevamente tenemos un índice mudo, \(r^{2}\), y si lo sustituimos por
	\(z\), ahora queda como algo que conocemos y sabemos calcular,

	\begin{align*}
		I^{2} & = \int_{0}^{\infty} \pi \odif{z} \mathrm{e}^{-a z^{2}},              \\
		      & = \dfrac{-\pi}{a} \biggl.\mathrm{e}^{-z}\biggr\rvert_{0}^{\infty},   \\
		      & = \dfrac{-\pi}{a} \bigl(\mathrm{e}^{-\infty} - \mathrm{e}^{0}\bigr), \\
		I^{2} & = \dfrac{\pi}{a},                                                    \\
		I     & = \sqrt{\dfrac{\pi}{a}}.
	\end{align*}

	Por lo tanto,

	\begin{empheq}[box=\mainresult]{equation}
		\int_{-\infty}^{\infty} \mathrm{e}^{-x^{2}}\odif{x} = \sqrt{\dfrac{\pi}{a}}.
		\label{eq:intNormalDist}
	\end{empheq}
\end{problema}
\end{document}
