\documentclass[../main.tex]{subfiles}

\begin{document}
\setproblem{16}
\begin{problema}[30]
	En clase encontramos que los electrones pueden tener momento angular intrínseco,
	el cual se denomina espín. En clase vimos que el espín puede ser representado por
	tres matrices:

	\begin{alignat*}{3}
		\sigma_{x} & = \begin{pNiceMatrix}
			               0 & 1 \\
			               1 & 0 \\
		               \end{pNiceMatrix},\qquad
		\sigma_{y} & {}={}                    & \begin{pNiceMatrix}
			                                        0 & -i \\
			                                        i & 0  \\
		                                        \end{pNiceMatrix},\qquad
		\sigma_{z} & {}={}                    &
		\begin{pNiceMatrix}
			1 & 0  \\
			0 & -1 \\
		\end{pNiceMatrix},
	\end{alignat*}

	conocidas como las matrices de Pauli. El problema consiste en verificar que las
	matrices de Pauli cumplen:

	\begin{enumerate}
		\item \(\sigma_{x}^{2} = 1,\sigma_{y}^{2} = 1 \) y\(\sigma_{z}^{2} = 1\)
		\item \([\sigma_{i}, \sigma_{j}] = 2i l\epsilon_{i j k} \sigma _{k}\) (asigne
		      valores para \(i, j\) y \(k\))
		\item \({\sigma_{i}, \sigma_{j}} = 2i\delta_{ij}\) (asigne
		      valores para \(i, j\) y \(k\)), donde \({A, B} \equiv AB + BA\)
		      se denomina anticonmutador.
	\end{enumerate}
\end{problema}
\end{document}
