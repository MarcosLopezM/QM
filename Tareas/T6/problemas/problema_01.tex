\documentclass[../main.tex]{subfiles}

\begin{document}
\setproblem{16}
\begin{problema}[30]
	En clase encontramos que los electrones pueden tener momento angular intrínseco,
	el cual se denomina espín. En clase vimos que el espín puede ser representado por
	tres matrices:

	\begin{alignat*}{3}
		\sigma_{x} & = \begin{pNiceMatrix}
			               0 & 1 \\
			               1 & 0 \\
		               \end{pNiceMatrix},\qquad
		\sigma_{y} & {}={}                    & \begin{pNiceMatrix}
			                                        0 & -i \\
			                                        i & 0  \\
		                                        \end{pNiceMatrix},\qquad
		\sigma_{z} & {}={}                    &
		\begin{pNiceMatrix}
			1 & 0  \\
			0 & -1 \\
		\end{pNiceMatrix},
	\end{alignat*}

	conocidas como las matrices de Pauli. El problema consiste en verificar que las
	matrices de Pauli cumplen:

	\begin{enumerate}
		\item \(\sigma_{x}^{2} = 1,\sigma_{y}^{2} = 1 \) y\(\sigma_{z}^{2} = 1\)
		\item \([\sigma_{i}, \sigma_{j}] = 2i l\epsilon_{i j k} \sigma _{k}\) (asigne
		      valores para \(i, j\) y \(k\))
		\item \({\sigma_{i}, \sigma_{j}} = 2i\delta_{ij}\) (asigne
		      valores para \(i, j\) y \(k\)), donde \({A, B} \equiv AB + BA\)
		      se denomina anticonmutador.
	\end{enumerate}

	\startsolution

	\section*{Inciso (a)}

	Calculamos el producto de cada matriz consigo misma, empezando por \(\sigma_{x}\),

	\begin{align*}
		\sigma_{x}^{2}             & = \begin{pNiceMatrix}
			                               0 & 1 \\
			                               1 & 0
		                               \end{pNiceMatrix}\begin{pNiceMatrix}
			                                                0 & 1 \\
			                                                1 & 0
		                                                \end{pNiceMatrix}, \\
		                           & = \begin{pNiceMatrix}
			                               0 + 1 & 0     \\
			                               0     & 1 + 0 \\
		                               \end{pNiceMatrix},                  \\
		                           & = \begin{pNiceMatrix}
			                               1 & 0 \\
			                               0 & 1 \\
		                               \end{pNiceMatrix},                  \\
		\Aboxedmain{\sigma_{x}^{2} & = 1.}
	\end{align*}

	Para \(\sigma_{y}^{2}\),

	\begin{align*}
		\sigma_{y}^{2}             & = \begin{pNiceMatrix}
			                               0 & -i \\
			                               i & 0
		                               \end{pNiceMatrix}\begin{pNiceMatrix}
			                                                0 & -i \\
			                                                i & 0
		                                                \end{pNiceMatrix}, \\
		                           & = \begin{pNiceMatrix}
			                               0 - i(i) & 0         \\
			                               0        & i(-i) + 0 \\
		                               \end{pNiceMatrix},                \\
		                           & = \begin{pNiceMatrix}
			                               1 & 0 \\
			                               0 & 1 \\
		                               \end{pNiceMatrix},                  \\
		\Aboxedmain{\sigma_{y}^{2} & = 1.}
	\end{align*}

	Finalmente, para \(\sigma_{z}^{2}\),

	\begin{align*}
		\sigma_{z}^{2}             & = \begin{pNiceMatrix}
			                               1 & 0  \\
			                               0 & -1
		                               \end{pNiceMatrix}\begin{pNiceMatrix}
			                                                1 & 0  \\
			                                                0 & -1
		                                                \end{pNiceMatrix}, \\
		                           & = \begin{pNiceMatrix}
			                               1 + 0 & 0         \\
			                               0     & 0 - 1(-1) \\
		                               \end{pNiceMatrix},                  \\
		\Aboxedmain{\sigma_{z}^{2} & = 1.}
	\end{align*}

	\section*{Inciso (b)}

	Calculamos los elementos necesarios para el conmutador,

	\begin{alignat*}{3}
		\sigma_{x}\sigma_{y} & = i\sigma_{z},\quad \sigma_{x}\sigma_{z}  & {}={} & -i\sigma_{y},\qquad \sigma_{y}\sigma_{z} & {}={} & i\sigma_{x}, \\
		\sigma_{y}\sigma_{x} & = -i\sigma_{z}\qquad \sigma_{z}\sigma_{x} & {}={} & i\sigma_{y},\qquad \sigma_{z}\sigma_{y}  & {}={} & -i\sigma_{x}
	\end{alignat*}

	Recordando la definición del conmutador,

	\begin{equation}
		[\sigma_{i}, \sigma_{j}] = \sigma_{i}\sigma_{j} - \sigma_{j}\sigma_{i}.
	\end{equation}

	Así notamos que

	\begin{alignat*}{2}
		[\sigma_{x}, \sigma_{y}] & = 2i\sigma_{z},\quad [\sigma_{y}, \sigma_{x}]  & {}={} & -2i\sigma_{z}, \\
		[\sigma_{x}, \sigma_{z}] & = -2i\sigma_{y},\quad [\sigma_{z}, \sigma_{x}] & {}={} & 2i\sigma_{y},  \\
		[\sigma_{y}, \sigma_{z}] & = 2i\sigma_{x},\quad [\sigma_{z}, \sigma_{y}]  & {}={} & -2i\sigma_{x}.
	\end{alignat*}

	Es decir, que cuando calculamos el conmutador permutado el signo cambia, por
	lo que podemos escribir los conmutadores anteriores como:

	\begin{empheq}[box = \mainresult]{equation*}
		[\sigma_{i}, \sigma_{j}] = 2i \epsilon_{ijk}\sigma_{k}.
	\end{empheq}

	\section*{Inciso (c)}

	Mientras que para el anticonmutador, de los productos de matrices de Pauli, que para
	\(i \neq j\),

	\begin{equation*}
		\bigl\lbrace \sigma_{i}, \sigma_{j} \bigr\rbrace = 0.
	\end{equation*}

	Sin embargo, para considerar los casos en lo que \(i = j\), de los resultados
	del inciso (a) y la definición del anticonmutador, tenemos que

	\begin{equation*}
		\bigl\lbrace \sigma_{i}, \sigma_{j} \bigr\rbrace = 2.
	\end{equation*}

	Así,

	\begin{empheq}[box = \mainresult]{equation*}
		\bigl\lbrace \sigma_{i}, \sigma_{j} \bigr\rbrace = 2 \delta_{ij}.
	\end{empheq}
\end{problema}
\end{document}
