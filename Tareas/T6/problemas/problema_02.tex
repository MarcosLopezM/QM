\documentclass[../main.tex]{subfiles}

\begin{document}
\setproblem{17}
\begin{problema}[60]
	En clase vimos que el momento angular puede ser expresado en coordenadas
	cartesianas como:

	\begin{equation*}
		\op{L}{k} = i \epsilon_{ijk} \op{r}{i}\op{p}{j}
	\end{equation*}

	donde \(\op{r}{i}\) es una componente del vector de posición y
	\(\op{p}{j} = \tfrac{\hbar}{i}\pdv{f}{r_{j}}\) es una componente del
	momentum. Teniendo en cuenta que en coordenadas esféricas las componentes
	del vector de posición son:

	\begin{alignat*}{3}
		x & = r \sin\theta\cos\phi,\qquad y & {}={}       & r\sin\theta\sin\phi,\qquad \text{y}\qquad
		z & {}={}                           & r\cos\theta
	\end{alignat*}

	muestre que

	\begin{enumerate}
		\item usando regla de la cadena se puede verificar la siguiente igualdad

		      \begin{equation*}
			      \op{L}{z} = -i\hbar\pdv{}{\phi}.
		      \end{equation*}
		\item Análogamente, con el uso de la regla de la cadena, verifique
		      que la igualdad

		      \begin{equation*}
			      \hbar\pdv{}{\theta} = i \Biggl(- \op{L}{x}\sin\phi + \op{L}{y}\cos\phi\Biggr).
		      \end{equation*}

		\item Verifique que la siguiente igualdad es correcta

		      \begin{equation*}
			      \dotprod{\op{\vect{r}}}{\op{\vect{L}}} = 0.
		      \end{equation*}
		\item Con las ecuaciones de los incisos anteriores, determine
		      \(\op{L}{x}\) y \(\op{L}{y}\) en coordenadas esféricas.

		\item Escriba la forma explícita de los operadores
		      \(\op{L}{\pm}\) en coordenadas esféricas.
		\item Usando la identidad \(\op{\vect{L}}{}{2} = \op{L}{+}\op{L}{-} + \op{L}{3}{2} -
		      \hbar\op{L}{3}\), corrobore que este operador en coordenadas esféricas tiene
		      la forma

		      \begin{equation*}
			      \op{\vect{L}}{}{2} = -\hbar^{2} \Biggl\lbrace \dfrac{1}{\sin^{2}\theta}
			      \pdv[order=2]{}{\phi} + \dfrac{1}{\sin\theta}\pdv{}{\theta}
			      \Biggl(\sin\theta\pdv{}{\theta}\Biggr)\Biggr\rbrace.
		      \end{equation*}
	\end{enumerate}

	\startsolution

	Tenemos que la regla de transformación general es

	\begin{equation}
		\pdv{}{r_{i}} = \pdv{}{x}\pdv{x}{r_{i}} + \pdv{}{y}\pdv{y}{r_{i}} +
		\pdv{}{z}\pdv{z}{r_{i}}.
		\label{eq:transformation}
	\end{equation}

	Entonces, para \(\pdv{}{\phi}\),

	\begin{align}
		\pdv{}{\phi}            & = \pdv{x}{\phi}\pdv{}{x} + \pdv{y}{\phi}\pdv{}{y} + \pdv{z}{\phi}\pdv{}{z},\nonumber \\
		                        & = \pdv*{(r\sin\theta\cos\phi)}{\phi}\pdv{}{x} +
		\pdv*{(r\sin\theta\sin\phi)}{\phi}\pdv{}{y} +
		\pdv*{(r\cos\theta)}{\phi}\pdv{}{z},\nonumber                                                                  \\
		                        & = -r\sin\theta\sin\phi\pdv{}{x} + r\sin\theta\cos\phi\pdv{}{y},\nonumber             \\
		                        & = -y\pdv{}{x} + x\pdv{}{y},\nonumber                                                 \\
		\Aboxedsec{\pdv{}{\phi} & = x\pdv{}{y} - y\pdv{}{x}.}\label{eq:pdvphi}
	\end{align}

	Si calculamos \(\op{L}{z}\) usando \(\op{L}{k} = \epsilon_{ijk}\op{r}{i}\op{p}{j}\),
	tenemos que

	\begin{align}
		\op{L}{z}            & = \epsilon_{ijz}\op{r}{i}\op{p}{j},\nonumber                            \\
		                     & = \epsilon_{xjz}\op{r}{x}\op{p}{j} + \epsilon_{yjz}\op{r}{y}\op{p}{j} +
		\epsilon_{zjz}\op{r}{z}\op{p}{j},\nonumber                                                     \\
		                     & = \epsilon_{xxz}\op{r}{x}\op{p}{x} + \epsilon_{xyz}\op{r}{x}\op{p}{y} +
		\epsilon_{xzz}\op{r}{x}\op{p}{z} + \epsilon_{yxz}\op{r}{y}\op{p}{x} +
		\epsilon_{yyz}\op{r}{y}\op{p}{y} + \epsilon_{yzz}\op{r}{y}\op{p}{z},\nonumber                  \\
		\Aboxedsec{\op{L}{z} & = \dfrac{\hbar}{i}\biggl(x\pdv{}{y} - y\pdv{}{x}\biggr).}
		\label{eq:Lz-cartesianas}
	\end{align}

	Sustituyendo la \zcref{eq:pdvphi} en la \zcref{eq:Lz-cartesianas}, obtenemos que

	\begin{align}
		\op{L}{z}             & = \dfrac{\hbar}{i}\pdv{}{\phi},\nonumber       \\
		\Aboxedmain{\op{L}{z} & = -i\hbar\pdv{}{\phi}.}\label{eq:Lz-spherical}
	\end{align}

	\pagebreak
	\section*{Inciso (b)}

	De la \zcref{eq:transformation} tenemos que el LHS queda como:

	\begin{align}
		\hbar \pdv{}{\theta} & = \hbar \Biggl[\pdv{x}{\theta}\pdv{}{x} +
		\pdv{y}{\theta}\pdv{}{y} + \pdv{z}{\theta}\pdv{}{z}\Biggr],\nonumber                  \\
		                     & = \hbar \Biggl[\pdv*{(r\sin\theta\cos\phi)}{\theta}\pdv{}{x} +
			\pdv*{(r\sin\theta\sin\phi)}{\theta}\pdv{}{y} +
		\pdv*{(r\cos\theta)}{\theta}\pdv{}{z}\Biggr],\nonumber                                \\
		                     & = \hbar \Biggl[r\cos\theta\cos\phi\pdv{}{x} +
		r\cos\theta\sin\phi\pdv{}{y} - r\sin\theta\pdv{}{z}\Biggr],\nonumber                  \\
		                     & = \hbar \Biggl[r\cos\theta \Biggl(\cos\phi\pdv{}{x} +
			\sin\phi\pdv{}{y}\Biggr) -
			r\sin\theta\pdv{}{z}\Biggr].
		\label{eq:hbar-pdvtheta}
	\end{align}

	Ahora, para el RHS, primero encontramos las expresiones para \(\op{L}{x}\) y
	\(\op{L}{y}\) en coordenadas cartesianas,

	\begin{align}
		\op{L}{x}            & = \epsilon_{ijx}\op{r}{i}\op{p}{j},\nonumber                                    \\
		                     & = \epsilon_{xjx}\op{r}{x}\op{p}{j} + \epsilon_{yjx}\op{r}{y}\op{p}{j} +
		\epsilon_{zjx}\op{r}{z}\op{p}{j},\nonumber                                                             \\
		                     & = \epsilon_{yzx}\op{r}{y}\op{p}{z} + \epsilon_{zyx}\op{r}{z}\op{p}{y},\nonumber \\
		                     & = \epsilon_{yzx}y \Bigl(\dfrac{\hbar}{i}\pdv{}{z}\Bigr) +
		\epsilon_{zyx}z \Bigl(\dfrac{\hbar}{i}\pdv{}{y}\Bigr),\nonumber                                        \\
		\Aboxedsec{\op{L}{x} & = \dfrac{\hbar}{i}\Biggl(y\pdv{}{z} - z\pdv{}{y}\Biggr)}.
		\label{eq:Lx-cartesian}
	\end{align}

	Análogamente,

	\begin{align}
		\op{L}{y}            & = \epsilon_{ijy}\op{r}{i}\op{p}{j},\nonumber                                    \\
		                     & = \epsilon_{xzy}\op{r}{x}\op{p}{z} + \epsilon_{zxy}\op{r}{z}\op{p}{x},\nonumber \\
		\Aboxedsec{\op{L}{y} & = \dfrac{\hbar}{i}\Bigl(z\pdv{}{x} - x\pdv{}{z}\Bigr).}
		\label{eq:Ly-cartesian}
	\end{align}

	\pagebreak
	Así,

	\begin{align*}
		RHS & = \hbar \Biggl[-\Biggl(r\sin\theta\sin\phi\pdv{}{z} - r\cos\theta\pdv{}{y}\Biggr)\sin\phi +
		\Biggl(r\cos\theta\pdv{}{x} - r\sin\theta\cos\phi\pdv{}{z}\Biggr)\cos\phi\Biggr],                 \\
		    & = \hbar \Biggl[-r\sin\theta(\sin^{2}\phi + \cos^{2}\phi)\pdv{}{z} +
		r\cos\theta \Bigl(\sin\phi\pdv{}{y} + \cos\phi\pdv{}{x}\Bigr)\Biggr],                             \\
		    & = \hbar \Biggl[r\cos\theta \Bigl(\cos\phi\pdv{}{y} + \sin\phi\pdv{}{y}\Bigr) -
			r\sin\theta\pdv{}{z}\Biggr].
	\end{align*}

	Por lo tanto, la igualdad se satisface.

	\section*{Inciso (c)}

	Desarrollando \(\dotprod{\op{\vect{r}}}{\op{\vect{L}}}\),

	\begin{align*}
		\dotprod{\op{\vect{r}}}{\op{\vect{L}}}             & = (x, y, z)\cdot (\op{L}{x}, \op{L}{y}, \op{L}{z}),                                         \\
		                                                   & = (r\sin\theta\cos\phi, r\sin\theta\sin\phi, r\cos\theta)(\op{L}{x}, \op{L}{y}, \op{L}{z}), \\
		                                                   & = r\sin\theta\cos\phi \Biggl[\dfrac{\hbar}{i}\Biggl(y\pdv{}{z} - z\pdv{}{y}\Biggr)\Biggr]
		+ r\sin\theta\sin\phi \Biggl[\dfrac{\hbar}{i}\Biggl(z\pdv{}{x} - x\pdv{}{z}\Biggr)\Biggr]
		+ r\cos\theta \Biggl[\dfrac{\hbar}{i}\Biggl(x\pdv{}{y} - y\pdv{}{x}\Biggr)\Biggr],                                                               \\
		                                                   & = \dfrac{\hbar}{i}\Biggl[
		\begin{multlined}[t]
			r^{2}\sin^{2}\theta\sin\phi\cos\phi\pdv{}{z} +
			r^{2}\sin\theta\sin\phi\cos\theta\pdv{}{x} +
			r^{2}\sin\theta\cos\phi\cos\theta\pdv{}{y}\\
			- r^{2}\sin\theta\cos\phi\cos\theta\pdv{}{y}
			- r^{2}\sin^{2}\theta\sin\phi\cos\phi\pdv{}{z}
			- r^{2}\cos\theta\sin\theta\sin\phi\pdv{}{x}\Biggr],
		\end{multlined}                                                                                              \\
		\Aboxedmain{\dotprod{\op{\vect{r}}}{\op{\vect{L}}} & = 0.}
	\end{align*}

	\section*{Inciso (d)}

	Del inciso (b) tenemos

	\begin{equation}
		- \op{L}{x}\sin\phi + \op{L}{y}\cos\phi = -i\hbar\pdv{}{\theta}.
		\label{eq:lx-ly-pdvtheta}
	\end{equation}

	Mientras que de (c),

	\begin{align*}
		r\sin\theta\cos\phi \op{L}{x} + r\sin\theta\sin\phi \op{L}{y} + r\cos\theta \op{L}{z} & = 0, \\
		\sin\theta\cos\phi \op{L}{x} + \sin\theta\sin\phi \op{L}{y} + \cos\phi \op{L}{z}      & = 0.
	\end{align*}

	Sustituyendo \(\op{L}{z} = -i \hbar \pdv{}{\phi}\),

	\begin{align}
		\sin\theta\sin\phi \op{L}{x} + \sin\theta\sin\phi \op{L}{y} & = -\cos\theta \Bigl(-i\hbar\pdv{}{\phi}\Bigr),\nonumber \\
		\sin\theta\cos\phi \op{L}{x} + \sin\theta\sin\phi \op{L}{y} & = i\hbar\cos\theta\pdv{}{\phi}.
		\label{eq:lxly-pdvphi}
	\end{align}

	Las \zcref{eq:lx-ly-pdvtheta,eq:lxly-pdvphi} forma un
	sistema de ecuaciones. Restando la \zcref{eq:lx-ly-pdvtheta} de la
	\zcref{eq:lxly-pdvphi}, multiplicando la primera por \(\sin\theta\sin\phi\) y
	la segunda por \(\cos\phi\),

	\begin{align*}
		\bigl(\sin\theta\cos^{2}\phi + \sin\theta\sin^{2}\phi\bigr)\op{L}{x} +
		\bigl(\sin\theta\sin\phi\cos\phi - \sin\theta\sin\phi\cos\phi\bigr)\op{L}{y} & =
		i\hbar \Biggl(\cos\theta\cos\phi\pdv{}{\phi} + \sin\theta\sin\phi\pdv{}{\theta}\Biggr),                                           \\
		\sin\theta \bigl(\cos^{2}\phi + \sin^{2}\phi\bigr)\op{L}{x}                  & = i\hbar \Biggl(\cos\phi\cos\theta\pdv{}{\phi} +
		\sin\theta\sin\phi\pdv{}{\theta}\Biggr),                                                                                          \\
		\sin\theta \op{L}{x}                                                         & = i\hbar \Biggl(\sin\theta\sin\phi\pdv{}{\theta} +
		\cos\theta\cos\phi\pdv{}{\phi}\Biggr).
	\end{align*}

	Por lo tanto,

	\begin{empheq}[box = \mainresult]{equation}
		\op{L}{x} = i\hbar \Biggl(\sin\phi\pdv{}{\theta} + \dfrac{\cos\phi}{\tan\theta}\pdv{}{\phi}\Biggr).
		\label{eq:lx-spherical}
	\end{empheq}

	Sustituyendo este resultado en la \zcref{eq:lx-ly-pdvtheta},

	\begin{align*}
		-\Biggl[i\hbar \Biggl(\sin\phi\pdv{}{\theta} + \dfrac{\cos\phi}{\tan\theta}\pdv{}{\phi}\Biggr)\Biggr]\sin\phi +
		\op{L}{y}\cos\phi & = -i\hbar\pdv{}{\theta},                           \\
		-i\hbar \Biggl[\sin^{2}\phi\pdv{}{\theta} + \dfrac{\sin\phi\cos\phi}{\tan\theta}\pdv{}{\phi}\Biggr] +
		\op{L}{y}\cos\phi & = -i\hbar\pdv{}{\theta},                           \\
		\op{L}{y}\cos\phi & = i\hbar \Biggl[(\sin^{2}\phi - 1)\pdv{}{\theta} +
			\dfrac{\sin\phi\cos\phi}{\tan\theta}\pdv{}{\phi}\Biggr].
	\end{align*}

	Por lo tanto,

	\begin{empheq}[box = \mainresult]{equation}
		\op{L}{y} = i\hbar \Biggl(-\cos\phi\pdv{}{\theta} + \dfrac{\sin\phi}{\tan\theta}\pdv{}{\phi}\Biggr).
		\label{eq:Ly-spherical}
	\end{empheq}

	\section*{Inciso (e)}

	Sabemos que la definición de los operadores de ascenso y descenso de
	momento angular es

	\begin{equation*}
		\op{L}{\pm} = \op{L}{x} \pm i \op{L}{y}.
	\end{equation*}

	Entonces,

	\begin{align*}
		\op{L}{+} & = i\hbar \Biggl(\sin\phi\pdv{}{\theta} + \dfrac{\cos\phi}{\tan\theta}\pdv{}{\phi}\Biggr) +
		i \Biggl(i \Biggl(-\cos\phi\pdv{}{\theta} + \dfrac{\sin\phi}{\tan\theta}\pdv{}{\phi}\Biggr)\Biggr),                     \\
		          & = \hbar \Biggl(\cos\phi\pdv{}{\theta} + i\sin\phi\pdv{}{\theta} -
		\dfrac{\sin\phi}{\tan\theta}\pdv{}{\phi} + i \dfrac{\cos\phi}{\tan\theta}\pdv{}{\phi}\Biggr),                           \\
		          & = \hbar \Bigl(\cos\phi + i\sin\phi\Bigr)\Biggl(\pdv{}{\theta} + i \dfrac{1}{\tan\theta}\pdv{}{\phi}\Biggr).
	\end{align*}

	Por lo tanto,

	\begin{empheq}[box = \mainresult]{equation}
		\op{L}{+} = \hbar \mathrm{e}^{i\phi}\Biggl(\pdv{}{\theta} + i \dfrac{1}{\tan\theta}\pdv{}{\phi}\Biggr).
		\label{eq:L+spherical}
	\end{empheq}

	Mientras que \(\op{L}{-}\),

	\begin{align*}
		\op{L}{-} & = i\hbar \Biggl(\sin\phi\pdv{}{\theta} + \dfrac{\cos\phi}{\tan\theta}\pdv{}{\phi}\Biggr) -
		i \Biggl(i\hbar \Biggl(-\cos\phi\pdv{}{\theta} + \dfrac{\sin\phi}{\tan\theta}\pdv{}{\phi}\Biggr)\Biggr),                 \\
		          & = \hbar \Biggl(-\cos\phi\pdv{}{\theta} + i\sin\phi\pdv{}{\theta} +
		\dfrac{\sin\phi}{\tan\theta}\pdv{}{\phi} + i\dfrac{\cos\phi}{\tan\theta}\pdv{}{\phi}\Biggr),                             \\
		          & = \hbar \Bigl(\cos\phi - i\sin\phi\Bigr)\Biggl(-\pdv{}{\theta} + i \dfrac{1}{\tan\theta}\pdv{}{\phi}\Biggr).
	\end{align*}

	Por lo tanto,

	\begin{empheq}[box = \mainresult]{equation}
		\op{L}{-} = \hbar\mathrm{e}^{-i\phi}\Biggl(-\pdv{}{\theta} + i \dfrac{1}{\tan\theta}\pdv{}{\phi}\Biggr).
		\label{eq:L-spherical}
	\end{empheq}

	\section*{Inciso (f)}

	Calculamos \(\op{L}{+}\op{L}{-}\) usando las \zcref{eq:L+spherical,eq:L-spherical}

	\begin{align}
		\op{L}{+}\op{L}{-} & = \hbar \mathrm{e}^{i\phi}\Biggl(\pdv{}{\theta} + i \dfrac{1}{\tan\theta}\pdv{}{\phi}\Biggr)\Biggl[-\mathrm{e}^{-i\phi}\pdv{}{\theta} + i\mathrm{e}^{-i\phi} \dfrac{1}{\tan\theta}\pdv{}{\phi}\Biggr],\nonumber \\
		                   & = \hbar^{2} \mathrm{e}^{i\phi}\Biggl\lbrace \begin{multlined}[t]
			                                                                 -\mathrm{e}^{-i\phi}\pdv[order=2]{}{\theta}
			                                                                 - i\csc^{2}\theta \mathrm{e}^{-i\phi}\pdv{}{\phi} +
			                                                                 i\dfrac{1}{\tan\theta}\mathrm{e}^{-i\phi}\pdv{}{\theta,\phi} \\
			                                                                 + \dfrac{i}{\tan\theta}\Biggl[
				                                                                 i \mathrm{e}^{-i\phi}\pdv{}{\theta} - \mathrm{e}^{-i\phi}\pdv{}{\phi,\theta} +
				                                                                 \dfrac{i}{\tan\theta}\Biggl(-i \mathrm{e}^{-i\phi}\pdv{}{\phi} + \mathrm{e}^{-i\phi}\pdv[order=2]{}{\phi} \Biggr)\Biggr]\Biggr\rbrace,
		                                                                 \end{multlined},\nonumber                                                                  \\
		                   & = \hbar^{2}\Biggl\lbrace  -\pdv[order=2]{}{\theta} -
		i\csc^{2}\theta\pdv{}{\phi} - \dfrac{1}{\tan\theta}\pdv{}{\theta} +
		\dfrac{i}{\tan^{2}\theta}\pdv{}{\phi} -
		\dfrac{1}{\tan^{2}\theta}\pdv[order=2]{}{\phi}\Biggr\rbrace,\nonumber                                                                                                                                                                \\
		                   & = -\hbar^{2}\Biggl\lbrace \pdv[order=2]{}{\theta} +
		i \Biggl(\dfrac{1}{\sin^{2}\theta} - \dfrac{\cos^{2}\theta}{\sin^{2}\theta}\Biggr)\pdv{}{\theta} +
		\dfrac{1}{\tan\theta}\pdv{}{\theta} +
		\dfrac{1}{\tan^{2}\theta}\pdv[order=2]{}{\phi}\Biggr\rbrace,\nonumber                                                                                                                                                                \\
		\op{L}{+}\op{L}{-} & = -\hbar^{2}\Biggl\lbrace \pdv[order=2]{}{\theta} + i\pdv{}{\phi} +
		\dfrac{1}{\tan\theta}\pdv{}{\theta} + \dfrac{1}{\tan^{2}\theta}\pdv[order=2]{}{\phi}\Biggr\rbrace.
		\label{eq:LxL-product}
	\end{align}

	De \(\op{L}{z} = -i\hbar\pdv{}{\phi}\), \(\op{L}{z}{2} = -h^{2}\pdv[order=2]{}{\phi}\)
	y de la \zcref{eq:LxL-product} tenemos

	\begin{align*}
		\op{\vect{L}}{}{2} & = -\hbar^{2}\Biggl\lbrace
		\pdv[order=2]{}{\theta} + i\pdv{}{\phi} +
		\dfrac{1}{\tan\theta}\pdv{}{\theta} + \dfrac{1}{\tan^{2}\theta}\pdv[order=2]{}{\phi}\Biggr\rbrace
		- \hbar^{2}\pdv[order=2]{}{\phi} + i\hbar^{2}\pdv{}{\phi},                               \\
		                   & = -\hbar^{2}\Biggl\lbrace \pdv[order=2]{}{\theta} + i\pdv{}{\phi} -
		i\pdv{}{\phi} + \dfrac{1}{\tan\theta}\pdv{}{\theta} +
		\unbr{\Biggl(1 + \dfrac{1}{\tan^{2}\theta}\Biggr)}_{(A)}\pdv[order=2]{}{\phi} \Biggr\rbrace.
	\end{align*}

	\pagebreak
	Desarrollando (A),

	\begin{align*}
		1 + \dfrac{1}{\tan^{2}\theta} & = \dfrac{\tan^{2}\theta + 1}{\tan^{2}\theta},                      \\
		                              & = \dfrac{\sec^{2}\theta}{\tan^{2}\theta},                          \\
		                              & = \dfrac{1}{\cos^{2}\theta}\dfrac{\cos^{2}\theta}{\sin^{2}\theta}, \\
		                              & = \dfrac{1}{\sin^{2}\theta}.
	\end{align*}

	Entonces,

	\begin{equation*}
		\op{\vect{L}}{}{2} = -\hbar^{2}\Biggl\lbrace \unbr{\pdv[order=2]{}{\theta} + \dfrac{1}{\tan\theta}\pdv{}{\theta}}_{(B)} +
		\dfrac{1}{\sin^{2}\theta}\pdv[order=2]{}{\phi} \Biggr\rbrace.
	\end{equation*}

	Escribiendo (B) como la derivada de un producto, i.e.

	\begin{align*}
		\dfrac{1}{\sin\theta}\pdv*{\Biggl(\sin\theta\pdv{}{\theta}\Biggr)}{\theta} & =
		\dfrac{1}{\sin\theta}\Biggl(\cos\theta\pdv{}{\theta} + \sin\theta\pdv[order=2]{}{\theta}\Biggr),                                              \\
		                                                                           & = \dfrac{1}{\tan\theta}\pdv{}{\theta} + \pdv[order=2]{}{\theta}.
	\end{align*}

	Por lo tanto,

	\begin{empheq}[box = \mainresult]{equation*}
		\op{\vect{L}}{}{2} = -\hbar^{2}\Biggl\lbrace \dfrac{1}{\sin^{2}\theta}\pdv[order=2]{}{\phi} +
		\dfrac{1}{\sin\theta}\pdv*{\Biggl(\sin\theta\pdv{}{\theta}\Biggr)}{\theta}
		\Biggr\rbrace.
	\end{empheq}
\end{problema}
\end{document}
