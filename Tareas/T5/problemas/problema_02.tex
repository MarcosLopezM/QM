\documentclass[../main.tex]{subfiles}

\begin{document}
\setproblem{15}
\begin{problema}
	Usando las relaciones \([r_{i}, p_{j}] - i \delta_{ij}\), mostrar que:

	\begin{equation*}
		[\op{\vect{L}}{}{2}, \op{L}{i}] = 0
	\end{equation*}

	con \(\op{\vect{L}} = \op{\vect{r}} \mul \op{\vect{p}}\) el momento
	angular y \( \op{L}{i}\) alguna componente de \(\op{\vect{L}}\).

	\startsolution

	Sabemos que \(\op{L}{i} = \epsilon_{ijk} \op{r}{j}\op{p}{k}\)
	y que \([\op{r}{i}, \op{p}{j}] = i\hbar \delta_{ij}\). Entonces,

	\begin{align}
		[\op{L}{}{2}, \op{L}{n}] & = \op{L}{m}[\epsilon_{mjk}\op{r}{j}\op{p}{k}, \epsilon_{nab}\op{r}{a}\op{p}{b}]
		+ [\epsilon_{mjk}\op{r}{j}\op{p}{k}, \epsilon_{nab}\op{r}{a}\op{p}{b}]\op{L}{m},\nonumber                  \\
		                         & = \op{L}{m}\epsilon_{mjk}\epsilon_{nab}
		[\op{r}{j}\op{p}{k}, \op{r}{a}\op{p}{b}]
		+ \epsilon_{mjk}\epsilon_{nab}
		[\op{r}{j}\op{p}{k}, \op{r}{a}\op{p}{b}]\op{L}{m}.\label{eq:commutator-L2-Ln}
	\end{align}

	Desarrollamos \(\epsilon_{mjk}\epsilon_{nab}[\op{r}{j}\op{p}{k}, \op{r}{a}\op{p}{b}]
	\),

	\begin{align*}
		\epsilon_{mjk}\epsilon_{nab}[\op{r}{j}\op{p}{k}, \op{r}{a}\op{p}{b}] & =
		\epsilon_{mjk}\epsilon_{nab} \Bigl(\op{r}{j}[\op{p}{k}, \op{r}{a}\op{p}{b}] +
		[\op{r}{j}, \op{r}{a}\op{p}{b}]\op{p}{k}\Bigr),                                                                                                            \\
		                                                                     & = \epsilon_{mjk}\epsilon_{nab}\Biggl(\op{r}{j}\Bigl([\op{p}{k}, \op{r}{a}]\op{p}{b}
			+ \op{r}{a}[\op{p}{k}, \op{p}{b}]\Bigr) +
		\Bigl([\op{r}{j}, \op{r}{a}]\op{p}{b} +
		\op{r}{a}[\op{r}{j}, \op{p}{b}]\Bigr)\op{p}{k}\Biggr),                                                                                                     \\
		                                                                     & = \epsilon_{mjk}\epsilon_{nab}\Bigl(\op{r}{j}(-i\hbar\delta_{ka})\op{p}{b} +
		\op{r}{a}(i\hbar\delta_{jb})\op{p}{k} \Bigr),                                                                                                              \\
		                                                                     & = i\hbar \Bigl(
		\epsilon_{mjk}\epsilon_{nab}\delta_{jb}\op{r}{a}\op{p}{k} - \epsilon_{mjk}\epsilon_{nab}\delta_{ka}\op{r}{j}\op{p}{b}\Bigr),                               \\
		                                                                     & = i\hbar \Bigl(
		\epsilon_{mbk}\epsilon_{nab}\op{r}{a}\op{p}{k} - \epsilon_{mja}\epsilon_{nab}\op{r}{j}\op{p}{b}\Bigr).
	\end{align*}

	Renombrando índices en el primer término de la expresión anterior:
	\(b \to a, k \to b, a \to j\),

	\begin{align}
		\epsilon_{mjk}\epsilon_{nab}[\op{r}{j}\op{p}{k}, \op{r}{a}\op{p}{b}] & =
		i\hbar \Bigl(\epsilon_{mab}\epsilon_{nja}\op{r}{j}\op{p}{b}
		- \epsilon_{mja}\epsilon_{nab}\op{r}{j}\op{p}{b}\Bigr),\nonumber                                                   \\
		                                                                     & = i\hbar \Bigl(\epsilon_{mab}\epsilon_{nja}
		- \epsilon_{mja}\epsilon_{nab}\Bigr)\op{r}{j}\op{p}{b}.
		\label{eq:commutator-rjpk-rapb}
	\end{align}

	\pagebreak
	Notemos que \(\epsilon_{mab} = - \epsilon_{mba}\) y \(\epsilon_{nab} = -\epsilon_{nba}\),
	así las relaciones \(\epsilon-\delta\),

	\begin{align*}
		\epsilon_{mba}\epsilon_{nja} & = \delta_{mn}\delta_{bj} - \delta_{mj}\delta_{bn}, \\
		\epsilon_{mja}\epsilon_{nba} & = \delta_{mn}\delta_{jb} - \delta_{mb}\delta_{jn}.
	\end{align*}

	Sustituyendo en la \zcref{eq:commutator-rjpk-rapb},

	\begin{align*}
		\epsilon_{mjk}\epsilon_{nab}[\op{r}{j}\op{p}{k}, \op{r}{a}\op{p}{b}] & =
		i\hbar \Bigl(-\delta_{mn}\delta_{bj} + \delta_{mj}\delta_{bn} + \delta_{mn}\delta_{jb} - \delta_{mb}\delta_{jn}\Bigr)\op{r}{j}\op{p}{b},                       \\
		                                                                     & = i\hbar \Bigl(\delta_{mj}\delta_{nb} - \delta_{mb}\delta_{nj}\Bigr)\op{r}{j}\op{p}{b}, \\
		                                                                     & = i\hbar\epsilon_{mnk}\epsilon_{jbk}\op{r}{j}\op{p}{b},                                 \\
		                                                                     & = i\hbar\epsilon_{mnk}\op{L}{k}.
	\end{align*}

	Entonces, la \zcref{eq:commutator-L2-Ln} queda como:

	\begin{equation*}
		[\op{L}{}{2}, \op{L}{n}] = \op{L}{m}\Bigl(i\hbar\epsilon_{mnk}\op{L}{k}\Bigr) +
		\Bigl(i\hbar\epsilon_{mnk}\op{L}{k}\Bigr)\op{L}{m}.
	\end{equation*}

	Intercambiando \(m \leftrightarrow k\) en el segundo término,

	\begin{equation*}
		[\op{L}{}{2}, \op{L}{n}] = i\hbar\epsilon_{mnk}\op{L}{m}\op{L}{k} +
		i\hbar\epsilon_{knm}\op{L}{m}\op{L}{k}.
	\end{equation*}

	Notemos que \(\epsilon_{kmn} = -\epsilon_{mnk}\),

	\begin{align*}
		[\op{L}{}{2}, \op{L}{n}] & = i\hbar\epsilon_{mnk}\Bigl(\op{L}{m}\op{L}{k} - \op{L}{m}\op{L}{k}\Bigr), \\
		                         & = i\hbar\epsilon_{mnk}(0).
	\end{align*}

	Por lo tanto,

	\begin{empheq}[box = \mainresult]{equation*}
		[\op{L}{}{2}, \op{L}{n}] = 0.
	\end{empheq}
\end{problema}
\end{document}
