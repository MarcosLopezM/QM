\documentclass[../main.tex]{subfiles}

\begin{document}
\setproblem{14}
\begin{problema}[10]
	Usando los operadores \(\op{L}{\pm}\), de ascenso y descenso de la
	proyección de momento angular a lo largo de la dirección \(z\),
	en clase quedó de tarea calcular las siguientes cantidades:

	\begin{enumerate}
		\item
		      \begin{equation*}
			      \matrixel{l^{\prime}m^{\prime}}{\op{L}{x}{2}}{lm}
		      \end{equation*}
		\item
		      \begin{equation*}
			      \matrixel{l^{\prime}m^{\prime}}{\op{L}{y}{2}}{lm}
		      \end{equation*}
		\item
		      \begin{equation*}
			      \matrixel{l^{\prime}m^{\prime}}{\op{L}{+}{2}}{lm}
		      \end{equation*}
		\item
		      \begin{equation*}
			      \matrixel{l^{\prime}m^{\prime}}{\op{L}{-}{2}}{lm}
		      \end{equation*}
	\end{enumerate}
\end{problema}
\end{document}
