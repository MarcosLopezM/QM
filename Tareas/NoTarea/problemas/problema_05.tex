\documentclass[../main.tex]{subfiles}

\begin{document}
\begin{problema}
	Considera un átomo de hidrógeno cuya función de onda es

	\begin{equation*}
		\Psi \bigl(\vect{r}\bigr) = A \phi_{200}\bigl(\vect{r}\bigr) +
		\dfrac{1}{\sqrt{5}}\phi_{311}\bigl(\vect{r}\bigr) +
		\dfrac{1}{\sqrt{3}}\phi_{422}\bigl(\vect{r}\bigr),
	\end{equation*}

	encuentra

	\begin{itemize}
		\item El valor de \(A\), de tal forma que el estado esté normalizado.
		\item Si se mide la energía, que valores son posibles. ¿Con qué probabilidad?
		\item La energía media del átomo.
	\end{itemize}
\end{problema}
\end{document}
