\documentclass[../main.tex]{subfiles}

\begin{document}
\begin{problema}
	Imagine que tiene un hamiltoniano de un sistema electrón-positrón, dado por

	\begin{equation*}
		\op{H} = \lambda \dotprod{\op{\vect{S}}{1}}{\op{\vect{S}}{2}} +
		\Biggl(\dfrac{eB}{mc}\Biggr)\Bigl(\op{S}{1z} - \op{S}{2z}\Bigr),
	\end{equation*}

	donde \(\lambda\) es un numero real y, \(\op{\vect{S}}{1}\) y \(\op{\vect{S}}{2}\), son
	los operadores de espín para el electron y el positron, respectivamente.

	\begin{itemize}
		\item Calcule el valor esperado de la energía del estado \(\ket{1, 0}\).
		\item  Repite el inciso anterior para \(\lambda = 0\), pero \(B \neq 0\).
		\item  Repite el primer inciso para \(\lambda \neq 0\), pero \(B = 0\).
	\end{itemize}
\end{problema}
\end{document}
