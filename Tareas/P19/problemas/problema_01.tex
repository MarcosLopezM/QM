\documentclass[../main.tex]{subfiles}

\begin{document}
\setproblem{18}
\begin{problema}
	Usando la ecuación radial:

	\begin{equation*}
		-\dfrac{\hbar^{2}}{2m}\odv[order = 2]{u_{n\ell}}{r} +
		\Biggl(\dfrac{1}{2}m\omega^{2}r^{2} + \dfrac{\ell(\ell + 1)}{2mr^{2}}\Biggr)u_{n\ell}
		= E u_{n\ell}.
	\end{equation*}

	Muestre que al proponer una solución de la forma:

	\begin{equation*}
		u = f(r)r^{\ell + 1}\mathrm{e}^{-m\omega r^{2}/2\hbar}.
	\end{equation*}

	la ecuación que satisface la función \(f(r)\) es:

	\begin{equation*}
		\odv[order = 2]{f}{r} + 2 \Biggl(\dfrac{\ell + 1}{r} - \dfrac{m\omega}{\hbar}r\Biggr)
		\odv{f}{r} + \Biggl[\dfrac{2mE}{\hbar^{2}} - (2\ell + 3)\dfrac{m\omega}{\hbar}\Biggr]f
		= 0.
	\end{equation*}

	\startsolution

	Calculamos la segunda derivada de la solución propuesta:

	\begin{align*}
		\odv{u}{r}            & = f^{\prime}r^{\ell + 1}\mathrm{e}^{-m\omega r^{2}/2\hbar} +
		(\ell + 1)f r^{\ell}\mathrm{e}^{-m\omega r^{2}/2\hbar} -
		\dfrac{m\omega}{\hbar}f r^{\ell + 2}\mathrm{e}^{-m\omega r^{2}/2\hbar},                                               \\
		\odv[order = 2]{u}{r} & = \begin{multlined}[t]
			                          f^{\prime\prime}r^{\ell + 1}\mathrm{e}^{-m\omega r^{2}/2\hbar}
			                          + (\ell + 1)f^{\prime} r^{\ell}\mathrm{e}^{-m\omega r^{2}/2\hbar} -
			                          \dfrac{m\omega}{\hbar}f^{\prime} r^{\ell + 2} \mathrm{e}^{-m\omega r^{2}/2\hbar} +
			                          (\ell + 1)f^{\prime} r^{\ell}\mathrm{e}^{-m\omega r^{2}/2\hbar}\\ +
			                          \ell(\ell + 1)f r^{\ell - 1}\mathrm{e}^{-m\omega r^{2}/2\hbar} +
			                          \dfrac{m\omega}{\hbar}\bigl(\ell + 1\bigr)f r^{\ell + 1}\mathrm{e}^{-m\omega r^{2}/2\hbar}
			                          - \dfrac{m\omega}{\hbar}f^{\prime}r^{\ell + 2}\mathrm{e}^{-m\omega r^{2}/2\hbar}\\ -
			                          \dfrac{m\omega}{\hbar}\bigl(\ell + 2\bigr)f r^{\ell + 1}\mathrm{e}^{-m\omega r^{2}/2\hbar}
			                          + \Biggl(\dfrac{m\omega}{\hbar}\Biggr)^{2} f r^{\ell + 3}\mathrm{e}^{-m\omega r^{2}/2\hbar}.
		                          \end{multlined}
	\end{align*}

	\pagebreak
	Simplificando y agrupando términos,

	\begin{equation*}
		\odv[order = 2]{u}{r} = \begin{multlined}[t]
			f^{\prime\prime}r^{\ell + 1}\mathrm{e}^{-m\omega r^{2}/2\hbar} +
			f^{\prime}\Biggl[\bigl(\ell + 1\bigr)\bigl(r^{\ell} + r^{\ell}\bigr) -
				\dfrac{m\omega}{\hbar}\bigl(r^{\ell + 2} + r^{\ell + 2}\bigr)
				\Biggr]\mathrm{e}^{-m\omega r^{2}/2\hbar}
			\\
			+ f \Biggl\lbrace \biggl(\dfrac{m\omega}{\hbar}\biggr)^{2}r^{\ell + 3} +
			\ell \bigl(\ell + 1\bigr)r^{\ell - 1} - \Biggl[\bigl(\ell + 1\bigr)\dfrac{m\omega}{\hbar}
				+ \bigl(\ell + 2\bigr)\dfrac{m\omega}{\hbar}\Biggr]r^{\ell + 1}\Biggr\rbrace \mathrm{e}^{-m\omega r^{2}/2\hbar}.
		\end{multlined}
	\end{equation*}

	Dividiendo por \(r^{\ell + 1}\mathrm{e}^{-m\omega r^{2}/2\hbar}\),

	\begin{equation}
		\dfrac{1}{r^{\ell + 1}\mathrm{e}^{-m\omega r^{2}/2\hbar}}\odv[order = 2]{u}{r} =
		f^{\prime\prime} +
		f^{\prime}\Biggl[\dfrac{2\bigl(\ell + 1\bigr)}{r} -
			\dfrac{2m\omega}{\hbar}r\Biggr]
		+ f \Biggl[\biggl(\dfrac{m\omega}{\hbar}\biggr)^{2}r^{2} +
		\dfrac{\ell \bigl(\ell + 1\bigr)}{r^{2}} -
		\bigl(2\ell +3\bigr)\dfrac{m\omega}{\hbar}\Biggr].
		\label{eq:second-dv-u}
	\end{equation}

	Regresando a la ecuación radial, sustituimos la solución propuesta y dividimos
	por \(r^{\ell + 1}\mathrm{e}^{-m\omega r^{2}/2\hbar}\). Luego multiplicamos por
	el inverso de \(-\tfrac{\hbar^{2}}{2m}\), i.e. \(-\tfrac{2m}{\hbar^{2}}\), para
	obtener una forma simplificada de la ecuación:

	\begin{equation}
		\dfrac{1}{r^{\ell + 1}\mathrm{e}^{-m\omega r^{2}/2\hbar}}\odv[order = 2]{u}{r} -
		\Biggl[\Biggl(\dfrac{m\omega}{\hbar}\Biggr)^{2}r^{2} +
		\dfrac{\ell \bigl(\ell + 1\bigr)}{r^{2}}
		- \dfrac{2mE}{\hbar^{2}}\Biggr]f = 0.
		\label{eq:radial-eq}
	\end{equation}

	Sustituyendo la \zcref{eq:second-dv-u} en la \zcref{eq:radial-eq},

	\begin{align*}
		f^{\prime\prime} + \Biggl[\dfrac{2(\ell + 1)}{r} - \dfrac{2m\omega}{\hbar}r\Biggr]f^{\prime} +
		\Biggl[\biggl(\dfrac{m\omega}{\hbar}\biggr)^{2}r^{2} + \dfrac{\ell \bigl(\ell + 1\bigr)}{r^{2}} - \dfrac{m\omega}{\hbar}\bigl(2\ell + 3\bigr)\Biggr]f -
		\Biggl[\biggl(\dfrac{m\omega}{\hbar}\biggr)^{2} +
			\dfrac{\ell \bigl(\ell + 1\bigr)}{r^{2}} -
		\dfrac{2mE}{\hbar^{2}}\Biggr]f                                               & = 0, \\
		\odv[order = 2]{f}{r} +
		2 \Biggl(\dfrac{\ell + 1}{r} - \dfrac{m\omega}{\hbar}r\Biggr)\odv{f}{r} +
		\Biggl[\biggl(\dfrac{m\omega^{2}}{\hbar}\biggr)^{2}\bigl(r^{2} - r^{2}\bigr) +
			\ell \bigl(\ell + 1\bigr)\Bigl(\dfrac{1}{r^{2}} - \dfrac{1}{r^{2}}\Bigr) +
		\dfrac{2mE}{\hbar^{2}} - \bigl(2\ell + 3\bigr)\dfrac{m\omega}{\hbar}\Biggr]f & = 0.
	\end{align*}

	Por lo tanto,

	\begin{empheq}[box = \mainresult]{equation*}
		\odv[order = 2]{f}{r} +
		2 \Biggl(\dfrac{\ell + 1}{r} - \dfrac{m\omega}{\hbar}r\Biggr)\odv{f}{r} +
		\Biggl[\dfrac{2mE}{\hbar^{2}} - \bigl(2\ell + 3\bigr)\dfrac{m\omega}{\hbar}\Biggr]f
		= 0.
	\end{empheq}
\end{problema}
\end{document}
