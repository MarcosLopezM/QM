\documentclass[../main.tex]{subfiles}

\begin{document}
\begin{problema}[40][Oscilador Armónico]
	Para los primeros estados de un oscilador armónico cuántico,
	encuentra, en términos de los operadores de ascenso y descenso,
	\(\hc{a}\) y \(\op{a}\) lo siguiente:

	\begin{enumerate}
		\item El valor esperado de \( \avg{\op{x}}\) y \(\avg{\avg{p}}\).
		\item El valor esperado de \(\avg{\op{x}{}{2}}\) y \(\avg{\op{p}{}{2}}\).
		\item Las desviaciones estándar \(\Delta r\) y \(\Delta p\).
		\item ¿Es \(\Delta r\) para \(n = 0\) igual que para \(n = 1\)?
		      Argumenta tu respuesta.
		\item ¿Es \(\Delta p\) para \(n = 0\) igual que para \(n = 1\)?
		      Argumenta tu respuesta.
		\item Encuentra el valor esperado de la energía cinética \(\avg{T}\)
		      y la energía potencial \(\avg{V}\). ¿Qué da su suma?
	\end{enumerate}

	\emph{Hint: Recuerda que los estados excitados se pueden encontrar a partir
		de los estados contiguos, \(\hc{a}\ket{n} = \sqrt{n + 1}\ket{n + 1},
		a \ket{n} = \sqrt{n} \ket{n - 1}\)}.
\end{problema}
\end{document}
