\documentclass[../main.tex]{subfiles}

\begin{document}
\begin{problema}[30][Barrera de Potencial]
	Considear una partícula de masa \(m\), que se mueve dentro de un potencial
	cuadrado infinito, cuya función de onda es

	\begin{equation*}
		\psi(x) = \dfrac{3}{\sqrt{30}}\phi_{0} + \dfrac{4}{\sqrt{30}}\phi_{1} +
		\dfrac{1}{\sqrt{6}}\phi_{4},
	\end{equation*}

	donde \(\phi_{n}\) es el \( n\)-ésimo estado excitado con energía
	\(E_{n} = \pi^{2}\hbar^{2}n^{2}/(2ma^{2})\).

	Encuentra:

	\begin{enumerate}
		\item La probabilidad de encontrar el sistema en el estado base.
		\item La energía promedio del sistema.
	\end{enumerate}
\end{problema}
\end{document}
