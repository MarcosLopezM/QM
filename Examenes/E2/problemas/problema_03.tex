\documentclass[../main.tex]{subfiles}

\begin{document}
\begin{problema}[20][Espinores de Dirac]
	Considera una partícula con spin \(s = \tfrac{1}{2}\) cuyo Hamiltoniano
	está dado por

	\begin{equation}
		H = \dfrac{\epsilon_{0}}{\hbar^{2}}\Biggl(\op{S}{x}{2} - \op{S}{y}{2}\Biggr)
		- \dfrac{\epsilon_{0}}{\hbar^{2}}\op{S}{z}{2},
	\end{equation}

	donde \(\epsilon_{0}\) es una constante con dimensiones de energía. Usando
	la representación matricial de \(\op{S}{i}\),

	\begin{enumerate}
		\item Encuentra la representación matricial del Hamiltoniano y diagonalízala para
		      encontrar los niveles de energía.
		\item Encuentra los eigenvectores y eigenvalores, e indica el grado de
		      degeneración.
	\end{enumerate}
\end{problema}
\end{document}
