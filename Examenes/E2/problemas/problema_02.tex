\documentclass[../main.tex]{subfiles}

\begin{document}
\begin{problema}[30][Momento angular]
	Considera una partícula con momento angular \(l = 1\).

	\begin{enumerate}
		\item Encuentra los eigenvalores y eigenvectores,
		      \(\ket{1, m_{y}}\) de \( L_{y}\).
		\item Expresa el estado \(\ket{1, m_{y} = 1}\) como una
		      superposición de eigenestados de \(L_{z}\).
		\item ¿Cuál es la probabilidad de medir \(m_{z} = 1\) cuando la
		      partícula está en el estado \(\ket{1, m_{y} = 1}\)? ¿Cuál sería
		      la probabilidad de medir \(m_{z} = 0\)?
	\end{enumerate}
\end{problema}
\end{document}
